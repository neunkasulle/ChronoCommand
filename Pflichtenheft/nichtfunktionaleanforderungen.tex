\section{Nichtfunktionale Anforderungen}

\subsection{Allgemeine Ziele}
\begin{requirements}
    \req{NF110} Die Navigation durch das \emph{Frontend} ist Intuitiv.
\end{requirements}

\subsection{Benutzbarkeit, Performance und Stabilität}
\begin{requirements}
    \req{NF220} Die Ladezeit des \emph{Frontends} liegt bei guter Internetverbindung unter 30 Sekunden.
    \req{NF230} Es wird auf Einblendungen, die die Benutzung des \emph{Frontends} einschränken verzichtet.
    \req{NF240} Das \emph{Frontend} läuft stabil, das bedeutet:
     \begin{itemize}
        \item Das \emph{Frontend} verhält sich zu jedem Zeitpunkt vorhersehbar.
        \item Alle Zustände und Übergänge sind zu jedem Zeitpunkt definiert.
        \item Unerwartete Eingaben werden abgefangen.
        \item Im Ein-Benutzer* Betrieb ist das \emph{Frontend} stabil (Eine \emph{Zeiterfassung} pro Benutzer*).
     \end{itemize}
     \req{NF250} Das \emph{Backend} läuft stabil, das bedeutet:
          \begin{itemize}
             \item Das \emph{Backend} verhält sich zu jedem Zeitpunkt vorhersehbar.
             \item Alle Zustände und übergänge sind zu jedem Zeitpunkt definiert.
             \item Unerwartete Eingaben werden abgefangen.
             \item Das \emph{Backend} verhält sich in folgenden Grenzen stabil:
             \begin{itemize}
                 \item Es werden höchtens 25 gleichzeitige Anfragen an das \emph{Backend} gestellt.
             \end{itemize}
          \end{itemize}
\end{requirements}

\subsection{Modularisierung in der Entwicklung}

\begin{requirements}
    \req{NF300} Um die Entwicklung und die Wartbarkeit des Produktes zu unterstützen, wird das Produkt in Module unterteilt.
    \begin{requirements}
        \req{NF310} Die Datenbank wird nicht modularisiert.
        \req{NF320} Das \emph{Backend} wird in folgende Module unterteilt:
            \begin{itemize}
                \item Kommunikation mit dem Frontend
                \item Kommunikation mit der Datenbank
                \item Benutzerauthentifizierung
                \item Benutzerverwaltung
                \item Gesetzeskonformität
                \item Benachrichtigungen
                \item Stundenzettelmanagement
            \end{itemize}
        \req{NF330} Das \emph{Frontend} wird in folgende Module unterteilt:
            \begin{itemize}
                \item Zeitmessung
                \item Benachrichtigungen
                \item Kalender
                \item Statistiken
                \item Stundenzettel Kern
            \end{itemize}
    \end{requirements}
\end{requirements}

\subsection{Qualität und Rechtliches}
\begin{requirements}
    \req{NF410} Die im Laufe des Projekts erstellte Artefakte sind gut
    \begin{itemize}
        \item zu warten (Einhaltung von Code Standards).
        \item zu erweitern (Objektorientierter, modularer Aufbau).
        \item dokumentiert.
        \item mit Testfällen abgedeckt (Unittests, Überdeckung > 75\%).
    \end{itemize}
    \req{NF420} Eine kommerzielle Veröffentlichung des Produkts ist möglich, u.a. gilt:
    \begin{itemize}
        \item benutzte Assets und Bibliotheken sind kommerziell nutzbar.
        \item es finden sich Hinweise auf die jeweiligen Urheber* und Lizenzen im Programm.
    \end{itemize}
    \req{NF430} \emph{Datenbankelemente} liegen mindestens in der dritten Normalform vor.
\end{requirements}
