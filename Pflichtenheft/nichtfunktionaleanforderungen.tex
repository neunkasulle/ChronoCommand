\section{Nichtfunktionale Anforderungen}

\subsection{Allgemeine Ziele}
\begin{requirements}
    \req{NF110} Die Navigation durch die Weboberfläche ist Intuitiv.
\end{requirements}

\subsection{Benutzbarkeit, Performance und Stabilität}
\begin{requirements}
    \req{NF220} Die Ladezeit des Frontends liegt bei guter Internetverbindung unter 30 Sekunden.
    \req{NF230} Es wird auf Einblendungen, die die Benutzung des Frontends einschränken verzichtet.
    \req{NF240} Das Frontend läuft stabil, das bedeutet:
     \begin{itemize}
        \item Das Frontend verhält sich zu jedem Zeitpunkt vorhersehbar.
        \item Alle Zustände und übergänge sind zu jedem Zeitpunkt definiert
        \item unerwartete Eingaben werden abgefangen.
        \item Im Ein-Benutzer betrieb ist das Frontend stabil \(Eine Zeiterfassung pro Nutzer\).
     \end{itemize}
     \req{NF250} Das backend läuft stabil, das bedeutet:
          \begin{itemize}
             \item Das Backend verhält sich zu jedem Zeitpunkt vorhersehbar.
             \item Alle Zustände und übergänge sind zu jedem Zeitpunkt definiert
             \item unerwartete Eingaben werden abgefangen.
             \item Das Backend verhält sich in folgenden Grenzen stabil:
                \begin{itemize}
                    \item Es werden höchtens 100 gleichzeitige Anfragen an das Backend gestellt.
                \end{itemize}
          \end{itemize}
\end{requirements}

\subsection{Qualität und Rechtliches}
\begin{requirements}
    \req{NF310} Die in Laufe des Projekts erstelle Artefakte sind gut
    \begin{itemize}
        \item zu warten \(Einhaltung von Code Standarts\).
        \item zu erweitern Ojektorientierter modularer Aufbau.
        \item dokumentiert.
        \item mit Testfällen abgedeckt \(Unittests, Überdeckung > 75\%\).
    \end{itemize}
    \req{NF330} Eine kommerzielle Veröffentlichung des Produkts ist möglich, u.a. gilt:
    \begin{itemize}
        \item benutzte Assets und Bibliotheken sind kommerziell nutzbar.
    	\item es finden sich Hinweise auf die jeweiligen Urheber und Lizenzen im Programm.
    \end{itemize}
\end{requirements}
