\section{Produktumgebung}

\subsection{Software}
\subsubsection{Software Backend}
\begin{itemize}
    \item Als Basis des Backends dient Java 8.
    \item Das Erstellen der benötigten Dateien übernimmt Maven.
    \item Als Interface für die Datenbank wird Hibernate ORM genutzt.
            Zur Datenbankanbindung wird dabei JDBC genutzt.
    \item Für die Authentifikation wird Apache Shiro verwendet.
    \item Das Generieren der PDF-Dateien aus den Stundenzetteln wird mithilfe von Apache PDFBox realisiert.
\end{itemize}

\subsubsection{Software Frontend}
\begin{itemize}
    \item Die Weboberfläche basiert auf HTML5 und wird mithilfe von VAADIN aus Java Code generiert.
\end{itemize}

\subsection{Hardware}
\subsubsection{Hardware User}
\begin{itemize}
    \item Da die Anwendung auf einer Web-Oberfläche läuft ist, muss die Hardware einen Webbrowser besitzen der HTML5 und JavaScript unterstützt.
\end{itemize}

\subsubsection{Hardware Server}
\begin{itemize}
    \item Der Server muss Java 8 installiert haben.
    \item Es wird ein Java Servlet Webserver benötigt.
    \item Es muss eine JDBC-kompatible Datenbank vorhanden sein.
\end{itemize}
