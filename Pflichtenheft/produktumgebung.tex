\section{Produktumgebung}

\subsection{Software}
\subsubsection{Software Backend}
\begin{itemize}
    \item Als Basis des \emph{Backends} dient Java 8.
    \item Das Erstellen der benötigten Dateien übernimmt Maven.
    \item Als Interface für die \emph{Datenbank} wird Hibernate ORM genutzt.
            Zur Datenbankanbindung wird dabei JDBC genutzt.
    \item Für die Authentifikation wird Apache Shiro verwendet.
    \item Das Generieren der PDF-Dateien aus den \emph{Stundenzetteln} wird mithilfe von Apache PDFBox realisiert.
\end{itemize}

\subsubsection{Software Frontend}
\begin{itemize}
    \item Das \emph{Frontend} ist eine auf HTML5 basierte Weboberfläche und wird mithilfe von VAADIN aus Java Code generiert.
\end{itemize}

\subsection{Hardware}
\subsubsection{Hardware Benutzer*}
\begin{itemize}
    \item Da die Anwendung auf einer Web-Oberfläche läuft ist, muss die Hardware einen Webbrowser besitzen der HTML5 und JavaScript unterstützt.
\end{itemize}

\subsubsection{Hardware Server}
\begin{itemize}
    \item Der Server muss Java 8 installiert haben.
    \item Es wird ein Java Servlet Webserver benötigt.
    \item Es muss eine JDBC-kompatible Datenbank vorhanden sein.
\end{itemize}
