\section{Globale Testfälle}

\subsection{Funktionssequenzen}
\begin{requirements}
	\req{T110} \emph{Benutzer} Meldet sich an
	\begin{requirements}
  			\req{T111}
  			    \textbf{Stand} Offenes Browserfenster. \\
  			    \textbf{Aktion} Der \emph{Benutzer} gibt chronoCommand.eu in die Browserzeile ein und bestätigt mit Enter. \\
  			    \textbf{Reaktion} Da kein Session Cookie geladen ist, wird der Anmeldescreen gezeigt.
  			\req{T112}
  			    \textbf{Stand} Die Anmeldeseite wurde geladen.\\
                \textbf{Aktion} Der \emph{Benutzer}
                    \begin{itemize}
                        \item gibt im Nutzernamen Feld seinen Nutzernamen an.
                        \item gibt im Passwort Feld sein Passwort an.
                        \item betätigt die Anmelden Schaltfläche.
                    \end{itemize}
                \begin{itemize}
			\item \textbf{Reaktion \emph{Benutzerdaten korrekt}} Der \emph{Benutzer} wird angemeldet.
			\item \textbf{Reaktion \emph{Benutzerdaten nicht korrekt}} Der \emph{Benutzer} bekommt eine Fehlermeldung.
                \end{itemize}

	\end{requirements}
	\req{T120} Angemeldeter \emph{Benutzer} startet/stoppt eine \emph{Zeiterfassung}.
	\begin{requirements}
	    \req{T121}
            \textbf{Stand} Der \emph{Benutzer} befindet sich angemeldet auf der \emph{Hauptseite}. \\
            \textbf{Aktion} Der \emph{Benutzer} klickt die Schaltfläche "`neue Zeiterfassung"' \\
            \textbf{Reaktion} Es werden die Elemente für eine neue \emph{Zeiterfassung} geladen.
            \item
                \textbf{Stand} Der \emph{Benutzer} sieht die Elemente für eine neue \emph{Zeiterfassung}. \\
                \textbf{Aktion} Der \emph{Benutzer}
                    \begin{itemize}
                        \item weist der \emph{Zeiterfassung} eine \emph{Kategorie} über das Textfeld "`Kategorie"' zu.
                        \item weist der \emph{Zeiterfassung} eine \emph{Tätigkeit} über das Textfeld "`Tätigkeit"' zu
                        \item startet die \emph{Zeiterfassung} über die Schaltfläche "`Start"'
                    \end{itemize}
                \textbf{Reaktion} Die \emph{Zeiterfassung} läuft.
        \req{T122}
            \textbf{Stand} Die \emph{Zeiterfassung} läuft. \\
            \textbf{Aktion} Der \emph{Benutzer} stoppt die \emph{Zeiterfassung} über die Schaltfläche "`Zeiterfassung stoppen"' \\
            \textbf{Reaktion}
                \begin{itemize}
                    \item Die \emph{Zeiterfassung} stoppt.
                    \item Die Erfasste Zeit wird in die erfassten Zeiten eingetragen.
                \end{itemize}
	\end{requirements}


	\req{T130} \emph{Benutzer} erhält \emph{Warnung} wegen nicht abgegebenem \emph{Studenzettel} \\
        \textbf{Stand} Der \emph{Benutzer} ist nicht eingeloggt. \\
        \textbf{Aktion} Der \emph{Benutzer} meldet sich an \emph{T110}. \\
        \textbf{Reaktion} Der \emph{Benutzer} erhält nach der Anmeldung ein Popup, dass ihn auffordert, den \emph{Stundenzettel} abzugeben.

	\req{T140} \emph{Betreuer} prüft neu \emph{abgebenen Stundenzettel}
	\begin{requirements}
	    \req{T141}
	        \textbf{Stand} Der \emph{Betreuer}
	            \begin{itemize}
	                \item ist angemeldet.
	                \item erhält über die Benachrichtigungs-Schaltfläche mittgeteilt das ein neuer \emph{Stundenzettel} abegeben wurde.
	            \end{itemize}
            \textbf{Aktion} Der \emph{Betreuer} klickt auf die Benachrichtigung. \\
            \textbf{Reaktion} Die entsprechende \emph{Stundenzettelseite} wird  geladen.
	    \req{T142}
	        \textbf{Stand} Die \emph{Stundenzettelseite} ist geladen. \\
            \textbf{Aktion}
                \begin{itemize}
                    \item Der \emph{Betreuer} überprüft den \emph{Stundenzettel}.
                    \item \textit{Stundenzettel Korrekt} Der \emph{Betreuer} klickt die Schaltfläche Stundenzettel geprüft.
                    \item \textit{Stundenzettel nicht Korrekt}  Der \emph{Betreuer} klickt die Schaltfläche Stundenzettel nicht Okay und kommentiert, was nicht korrekt ist.
                \end{itemize}
            \textbf{Reaktion}
                \begin{itemize}
                    \item Die \emph{Stundenzettelseite} wird geschlossen der \emph{Betreuer} wird zur \emph{Hauptseite} zurückgeleitet
                    \item \textit{Stundenzettel Korrekt} Der \emph{Stundenzettel} ist als \emph{geprüft} markiert und der \emph{Administrator} wird darüber benachrichtigt
                    \item \textit{Stundenzettel nicht Korrekt} Der \emph{Stundenzettel} wird als \emph{nicht abgegben} markiert und der \emph{Benutzer} darüber benachrichtigt
                \end{itemize}
	\end{requirements}

	\req{T150} \emph{Benutzer} editiert die \emph{Tätigkeit} in einem \emph{nicht abgegebenem} \emph{Stundenzettel}
	\begin{requirements}
	    \req{T151}
	        \textbf{Stand} \emph{Benutzer} befindet sich auf seiner \emph{Hauptseite} mit den Auflistungen der eingetragenen Zeiten. \\
	        \textbf{Aktion} Der \emph{Benutzer} klickt dem Editieren Button bei einer eingetragenen Zeit.\\
            \textbf{Reaktion} Die Elemente der eingetragenen Zeit werden editierbar.
        \req{T152}
        \textbf{Stand} Die Elemente der eingetragenen Zeit sind editierbar. \\
        \textbf{Aktion} Der \emph{Benutzer}
            \begin{itemize}
                \item klickt in das Textfeld in dem die \emph{Tätigkeit} vermerkt wird.
                \item ändert die dort eingetragene \emph{Tätigkeit}.
                \item klickt auf die Speichern Schaltfläche.
            \end{itemize}
            \textbf{Reaktion} Die eingetragene \emph{Tätigkeit} ist damit geändert
    \end{requirements}

    \req{T160} Der \emph{Benutzer} wählt die Seite an und besitz ein gültiges Session Cookie. \\
        \textbf{Stand} Offenes Browserfenster. \\
        \textbf{Aktion} Der \emph{Benutzer} gibt chronoCommand.eu in die Browserzeile ein in betätigt Enter. \\
        \textbf{Reaktion} Da Session Cookie geladen, wird die \emph{Benutzer Hauptseite} gezeigt.

    \req{T170} Der \emph{Benutzer} verletzt gesetzliche Vorgaben. \\
        \textbf{Stand} \emph{Zeiterfassung} läuft. \\
        \textbf{Aktion} Die \emph{Zeiterfassung} überschreitet ein gesetzes Limit. \\
        \textbf{Reaktion}
            \begin{itemize}
                \item Der \emph{Benutzer} erhält eine \emph{Warnung}, dass er gesetzliche Vorgaben verletzt.
                \item Die \emph{Zeiterfassung} stoppt.
            \end{itemize}

    \req{T180} \emph{Betreuer} sieht \emph{Stundendaten} ein. \\
        \textbf{Stand} Der \emph{Betreuer} befindet sich auf der \emph{Betreuer Hauptseite}. \\
        \textbf{Aktion} Der \emph{Betreuer} klickt die Schaltfläche "`Stundenzettel einsehen"' \\
        \textbf{Reaktion} Der \emph{Betreuer} erhält die \emph{Stundendaten} der ihm zugewiesenen \emph{Benutzer}.

\end{requirements}
