\section{Globale Testfälle}

\subsection{Funktionssequenzen}
\begin{requirements}
	\req{T110} \em{Benutzer} Meldet sich an
	\begin{requirements}
  			\req{T111}
  			    \textbf{Stand} Offenes Browserfenster. \\
  			    \textbf{Aktion} Der \em{Benutzer} gibt chronoCommand.eu in die Browserzeile ein und bestätigt mit Enter. \\
  			    \textbf{Reaktion} Da kein Session Cookie geladen ist, wird der Anmeldescreen gezeigt.
  			\req{T112}
  			    \textbf{Stand} Die Anmeldeseite wurde geladen.\\
                \textbf{Aktion} Der \em{Benutzer}
                    \begin{itemize}
                        \item gibt im Nutzernamen Feld seinen Nutzernamen an.
                        \item gibt im Passwort Feld sein Passwort an.
                        \item betätigt die Anmelden Schaltfläche.
                    \end{itemize}
                \begin{itemize}
			\item \textbf{Reaktion \em{Benutzerdaten korrekt}} Der \em{Benutzer} wird angemeldet.
			\item \textbf{Reaktion \em{Benutzerdaten nicht korrekt}} Der \em{Benutzer} bekommt eine Fehlermeldung.
                \end{itemize}

	\end{requirements}
	\req{T120} Angemeldeter \em{Benutzer} startet/stoppt eine \em{Zeiterfassung}.
	\begin{requirements}
	    \req{T121}
            \textbf{Stand} Der \em{Benutzer} befindet sich angemeldet auf der \em{Hauptseite}. \\
            \textbf{Aktion} Der \em{Benutzer} klickt die Schaltfläche "`neue Zeiterfassung"' \\
            \textbf{Reaktion} Es werden die Elemente für eine neue \em{Zeiterfassung} geladen.
            \item
                \textbf{Stand} Der \em{Benutzer} sieht die Elemente für eine neue \em{Zeiterfassung}. \\
                \textbf{Aktion} Der \em{Benutzer}
                    \begin{itemize}
                        \item weist der \em{Zeiterfassung} eine \em{Kategorie} über das Textfeld "`Kategorie"' zu.
                        \item weist der \em{Zeiterfassung} eine \em{Tätigkeit} über das Textfeld "`Tätigkeit"' zu
                        \item startet die \em{Zeiterfassung} über die Schaltfläche "`Start"'
                    \end{itemize}
                \textbf{Reaktion} Die \em{Zeiterfassung} läuft.
        \req{T122}
            \textbf{Stand} Die \em{Zeiterfassung} läuft. \\
            \textbf{Aktion} Der \em{Benutzer} stoppt die \em{Zeiterfassung} über die Schaltfläche "`Zeiterfassung stoppen"' \\
            \textbf{Reaktion}
                \begin{itemize}
                    \item Die \em{Zeiterfassung} stoppt.
                    \item Die Erfasste Zeit wird in die erfassten Zeiten eingetragen.
                \end{itemize}
	\end{requirements}


	\req{T130} \em{Benutzer} erhält \em{Warnung} wegen nicht abgegebenem \em{Studenzettel} \\
        \textbf{Stand} Der \em{Benutzer} ist nicht eingeloggt. \\
        \textbf{Aktion} Der \em{Benutzer} meldet sich an \em{T110}. \\
        \textbf{Reaktion} Der \em{Benutzer} erhält nach der Anmeldung ein Popup, dass ihn auffordert, den \em{Stundenzettel} abzugeben.

	\req{T140} \em{Betreuer} prüft neu \em{abgebenen Stundenzettel}
	\begin{requirements}
	    \req{T141}
	        \textbf{Stand} Der \em{Betreuer}
	            \begin{itemize}
	                \item ist angemeldet.
	                \item erhält über die Benachrichtigungs-Schaltfläche mittgeteilt das ein neuer \em{Stundenzettel} abegeben wurde.
	            \end{itemize}
            \textbf{Aktion} Der \em{Betreuer} klickt auf die Benachrichtigung. \\
            \textbf{Reaktion} Die entsprechende \em{Stundenzettelseite} wird  geladen.
	    \req{T142}
	        \textbf{Stand} Die \em{Stundenzettelseite} ist geladen. \\
            \textbf{Aktion}
                \begin{itemize}
                    \item Der \em{Betreuer} überprüft den \em{Stundenzettel}.
                    \item \textit{Stundenzettel Korrekt} Der \em{Betreuer} klickt die Schaltfläche Stundenzettel geprüft.
                    \item \textit{Stundenzettel nicht Korrekt}  Der \em{Betreuer} klickt die Schaltfläche Stundenzettel nicht Okay und kommentiert, was nicht korrekt ist.
                \end{itemize}
            \textbf{Reaktion}
                \begin{itemize}
                    \item Die \em{Stundenzettelseite} wird geschlossen der \em{Betreuer} wird zur \em{Hauptseite} zurückgeleitet
                    \item \textit{Stundenzettel Korrekt} Der \em{Stundenzettel} ist als \em{geprüft} markiert und der \em{Administrator} wird darüber benachrichtigt
                    \item \textit{Stundenzettel nicht Korrekt} Der \em{Stundenzettel} wird als \em{nicht abgegben} markiert und der \em{Benutzer} darüber benachrichtigt
                \end{itemize}
	\end{requirements}

	\req{T150} \em{Benutzer} editiert die \em{Tätigkeit} in einem \em{nicht abgegebenem} \em{Stundenzettel}
	\begin{requirements}
	    \req{T151}
	        \textbf{Stand} \em{Benutzer} befindet sich auf seiner \em{Hauptseite} mit den Auflistungen der eingetragenen Zeiten. \\
	        \textbf{Aktion} Der \em{Benutzer} klickt dem Editieren Button bei einer eingetragenen Zeit.\\
            \textbf{Reaktion} Die Elemente der eingetragenen Zeit werden editierbar.
        \req{T152}
        \textbf{Stand} Die Elemente der eingetragenen Zeit sind editierbar. \\
        \textbf{Aktion} Der \em{Benutzer}
            \begin{itemize}
                \item klickt in das Textfeld in dem die \em{Tätigkeit} vermerkt wird.
                \item ändert die dort eingetragene \em{Tätigkeit}.
                \item klickt auf die Speichern Schaltfläche.
            \end{itemize}
            \textbf{Reaktion} Die eingetragene \em{Tätigkeit} ist damit geändert
    \end{requirements}

    \req{T160} Der \em{Benutzer} wählt die Seite an und besitz ein gültiges Session Cookie. \\
        \textbf{Stand} Offenes Browserfenster. \\
        \textbf{Aktion} Der \em{Benutzer} gibt chronoCommand.eu in die Browserzeile ein in betätigt Enter. \\
        \textbf{Reaktion} Da Session Cookie geladen, wird die \em{Benutzer Hauptseite} gezeigt.

    \req{T170} Der \em{Benutzer} verletzt gesetzliche Vorgaben. \\
        \textbf{Stand} \em{Zeiterfassung} läuft. \\
        \textbf{Aktion} Die \em{Zeiterfassung} überschreitet ein gesetzes Limit. \\
        \textbf{Reaktion}
            \begin{itemize}
                \item Der \em{Benutzer} erhält eine \em{Warnung}, dass er gesetzliche Vorgaben verletzt.
                \item Die \em{Zeiterfassung} stoppt.
            \end{itemize}

    \req{T180} \em{Betreuer} sieht \em{Stundendaten} ein. \\
        \textbf{Stand} Der \em{Betreuer} befindet sich auf der \em{Betreuer Hauptseite}. \\
        \textbf{Aktion} Der \em{Betreuer} klickt die Schaltfläche "`Stundenzettel einsehen"' \\
        \textbf{Reaktion} Der \em{Betreuer} erhält die \em{Stundendaten} der ihm zugewiesenen \em{Benutzer}.

\end{requirements}
