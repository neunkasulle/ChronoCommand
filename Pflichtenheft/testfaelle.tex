\section{Globale Testfälle}

\subsection{Funktionssequenzen}
\begin{requirements}
	\req{T110} Benutzer Meldet sich an
	\begin{itemize}
  			\item
  			    \textbf{Stand} Offenes Browserfenster. \\
  			    \textbf{Aktion} Der Benutzer gibt chronoCommand.eu in die Browserzeile ein in betätigt Enter. \\
  			    \textbf{Reaktion} Da Kein Session Cookie geladen, wird der Anmeldescreen gezeigt.
  			\item
  			    \textbf{Stand} Der Anmeldeseite wurde geladen.\\
                \textbf{Aktion}
                    \begin{itemize} Der Benutzer
                        \item gibt im Nutzernamen Feld seinen Nutzernamen an.
                        \item gibt im Passwort Feld sein Passwort an.
                        \item betätigt die Anmelden Schaltfläche.
                    \end{itemize}
                \begin{itemize}
                    \textbf{Reaktion\[Benutzerdaten Korrekt\]} Der Benutzer wird Angemeldet.
                    \textbf{Reaktion\[Benutzerdaten Nicht Korrekt\]} Der Benutzer bekommt eine Fehlermeldung.
                \end{itemize}

	\end{itemize}
	\req{T120} Angemeldeter Benutzer startet/stoppt eine Zeiterfassung.
	\begin{itemize}
	    \item
            \textbf{Stand} Der Benutzer befindet sich angemeldet auf der Hauptseite. \\
            \textbf{Aktion} Der Benutzer klickt die Schaltfläche "`neue Zeiterfassung"' \\
            \textbf{Reaktion} Es werden die Elemente für eine neue Zeiterfassung geladen.
            \item
                \textbf{Stand} Der Benutzer sieht die Elemente für eine neue Zeiterfassung. \\
                \textbf{Aktion}
                    \begin{itemize} Der Benutzer
                        \item weist der Zeiterfassung eine Kategorie über das Textfeld "`Kategorie"' zu.
                        \item weist der Zeiterfassung eine Tätigkeit über das Textfeld "`Tätigkeit"' zu
                        \item startet die Zeiterfassung über die Schaltfläche "`Start"'
                    \end{itemize}
                \textbf{Reaktion} Die Zeiterfassung läuft.
        \item
            \textbf{Stand} Die Zeiterfassung läuft. \\
            \textbf{Aktion} Der Benutzer stoppt die Zeiterfassung über die Schaltfläche "`Zeiterfassung stoppen"' \\
            \textbf{Reaktion}
                \begin{itemize}
                    \item Die Zeiterfassung Stoppt.
                    \item Die Erfasste Zeit wird in die erfassten Zeiten eingetragen.
                \end{itemize}
	\end {itemize}


	\req{T130} Admin sammelt Stundendaten ein
            \textbf{Stand} Der Admin befindet sich auf der Admin Hauptseite. \\
            \textbf{Aktion} Der Admin klickt die Schaltfläche "`Stundenzettel einsammeln"'' \\
            \textbf{Reaktion}
                \begin{itemize} Der Admin erhält
                    \item Informationen über den Erfolg des Einsammelns
                    \item die Stundendaten der Benutzer
                \end{itemize}

	\req{T130} Betreuer sieht Stundendaten ein
	        \textbf{Stand} Der Betreuer befindet sich auf der Betreuer Hauptseite. \\
            \textbf{Aktion} Der Betreuer klickt die Schaltfläche "`Stundenzettel einsehen"' \\
            \textbf{Reaktion} Der Betreuer erhält die Stundendaten der ihm zugewiesenen Benutzer.


	\req{T140} Benutzer erhält Warnung wegen nicht abgegebenem Studenzettel
        \textbf{Stand} Der Benutzer ist nicht eingeloggt. \\
        \textbf{Aktion} Der Benutzer meldet sich an \textbf{T110}. \\
        \textbf{Reaktion} Der Benutzer erhält nach der Anmeldung ein Popup, dass ihn auffordert, den Stundenzettel abzugeben.

	\req{T150} Betreuer prüft neu abgebenen Stundenzettel
	\begin{itemize}
	    \item
	        \textbf{Stand}
	            \begin{itemize} Der Betreuer
	                \item ist angemeldet.
	                \item erhält über die Benachrichtigungs-Schaltfläche mittgeteilt das ein neuer Stundenzettel abegeben wurde.
	            \end{itemize}
            \textbf{Aktion} Der Betreuer klickt auf die Benachrichtigung. \\
            \textbf{Reaktion} Die entsprechende Stundenzettelseite wird  geladen.
	    \item
	        \textbf{Stand} Die Stundenzettelseite ist geladen. \\
            \textbf{Aktion}
                \begin{itemize}
                    \item Der Betreuer überprüft den Stundenzettel.
                    \item \textit{Stundenzettel Korrekt} Der Betreuer klickt die Schaltfläche Stundenzettel Okay.
                    \item \textit{Stundenzettel nicht Korrekt}  Der Betreuer klickt die Schaltfläche Stundenzettel nicht Okay.
                \end{itemize}
            \textbf{Reaktion}
                \begin{itemize}
                    \item Die Stundenzettel Seite wird geschlossen der Betreuer wird zur Hauptseite zurückgeleitet
                    \item \textit{Stundenzettel Korrekt} Der Stundenzettel ist als Okay markiert und der Admin wird darüber benachrichtigt
                    \item \textit{Stundenzettel nicht Korrekt} Der Stundenzettel wird als nicht abgegben markiert und der Benutzer darüber benachrichtigt
                \end{itemize}
	\end{itemize}

	\req{T160} Benutzer editiert die Tätigkeit in einem nicht abgegebenem Stundenzettel
	\begin{itemize}
	    \item
	        \textbf{Stand} Benutzer befindet sich auf seiner Hauptseite mit den auflistungen der eingetragenen Zeiten. \\
	        \textbf{Aktion} Der Benutzer klickt dem Editieren Button bei einer eingetragenen Zeit.\\
            \textbf{Reaktion} Die Elemente der eingetragenen Zeit werden editierbar.
        \item
        \textbf{Stand} Die Elemente der eingetragenen Zeit sind editierbar.\\
        \textbf{Aktion}
            \begin{itemize} Der Benutzer
                \item klickt in das Textfeld in dem die Tätigkeit vermerkt wird.
                \item ändert die dort eingetragene Tätigkeit.
                \item klickt auf die Speichern Schaltfläche.
            \end{itemize}
            \textbf{Reaktion} Die eingetragene Tätigkeit ist damit geändert
    \end{itemize}

    \req{T110} Der Benutzer wählt die Seite an und besitz ein gültiges Session Cookies
        \textbf{Stand} Offenes Browserfenster. \\
        \textbf{Aktion} Der Benutzer gibt chronoCommand.eu in die Browserzeile ein in betätigt Enter. \\
        \textbf{Reaktion} Da Session Cookie geladen, wird die Benutzer Hauptseite gezeigt.

\end{requirements}
