\section{Globale Testfälle}

\subsection{Funktionssequenzen}
\begin{requirements}

	\req{T110} Benutzer Meldet sich an
	\begin{itemize}
  			\item Der Benutzer wählt in seinem Webbrowser die Website der Zeiterfassung an
  			\item Da der Benutzer keinen Session Cookie besitzt wird ihm ein Anmeldescreen gezeigt
  			\item Der Benutzer gibt dort seine Email an
  			\item Der Benutzer gibt dort sein Passwort an
  			\item Der Benutzer klickt auf die Schaltfläche "Anmelden"
  			\item Wenn Email und Passwort korrekt, hat sich der Benutzer damit erfolgreich angemeldet
    \end{itemize}

    \req{T120} Angemeldeter Benutzer startet/stoppt eine Zeiterfassung
    \begin{itemize}
        \item Der Benutzer befindet sich auf der Hauptseite
        \item Der Benutzer klickt die Schaltfläche "neue Zeiterfassung"
        \item Der Benutzer weist der Zeiterfassung eine Kategorie über die Schaltfläche "Kategorie" zu
        \item Der Benutzer weist der Zeiterfassung eine Tätogkeit über das Textfeld "Tätogkeit" zu
        \item Der Benutzer startet die Zeiterfassung über die Schaltfläche "Start"
        \item Der Benutzer stoppt die Zeiterfassung über die Schaltfläche "Zeiterfassung stoppen"
    \end {itemize}

    \reg{T130} Admin sammelt Stundendaten ein
    \begin{itemize}
            \item Der Admin befindet sich auf der Admin Hauptseite
            \item Der Admin klickt die Schaltfläche "Stundenzettel einsammeln"
            \item Der Admin erhält Informationen über den Erfolg des einsammelns
            \item Der Admin erhält die Stundendaten der Benutzer
        \end {itemize}

    \reg{T130} Betreuer sieht Stundendaten ein
        \begin{itemize}
                \item Der Betreuer befindet sich auf der Betreuer Hauptseite
                \item Der Betreuer klickt die Schaltfläche "Stundenzettel einsehen"
                \item Der Betreuer erhält die Stundendaten der ihm zugewiesenen Benutzer
            \end {itemize}

    \reg{T140} Benutzer erhält Warnung wegen nicht abgegebenem Studenzettel
            \begin{itemize}
                    \item Der Benutzer meldet sich an
                    \item Der Benutzer erhält nach der Anmeldung ein Popup, dass ihn auffordert, den Stundenzettel abzugeben.
                    \item Der Benutzer kann im Popup über die Schaltfläche "zum Stundenzettel" auf die Stundenzettel seite wechseln und diesen dort als abzugeben markieren.
                \end {itemize}

\end{requirements}