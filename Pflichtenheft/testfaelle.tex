\section{Globale Testfälle}

\subsection{Funktionssequenzen}
\begin{requirements}
	\req{T110} \emph{Benutzer*} meldet sich an.
	\begin{requirements}
  			\req{T111}
  			    \textbf{Stand} Offenes Browserfenster. \\
  			    \textbf{Aktion} Der \emph{Benutzer*} gibt die lokale URL der ChronoCommand-Instanz in die Browserzeile ein und bestätigt mit Enter. \\
  			    \textbf{Reaktion} Da kein Session Cookie geladen ist, wird der Anmeldescreen gezeigt.
  			\req{T112}
  			    \textbf{Stand} Die Anmeldeseite wurde geladen.\\
                \textbf{Aktion} Der \emph{Benutzer*}
                    \begin{itemize}
                        \item gibt im Benutzer*namen Feld seinen Benutzer*namen an.
                        \item gibt im Passwort Feld sein Passwort an.
                        \item betätigt die Anmelden Schaltfläche.
                    \end{itemize}
                \begin{itemize}
			\item \textbf{Reaktion \emph{Benutzer*daten korrekt}} Der \emph{Benutzer*} wird angemeldet.
			\item \textbf{Reaktion \emph{Benutzer*daten nicht korrekt}} Der \emph{Benutzer*} bekommt eine Fehlermeldung.
                \end{itemize}

	\end{requirements}
	\req{T120} Angemeldeter \emph{Benutzer*} startet/stoppt eine \emph{Zeiterfassung}.
	\begin{requirements}
	    \req{T121}
            \textbf{Stand} Der \emph{Benutzer*} befindet sich angemeldet auf der \emph{Hauptseite}. \\
            \textbf{Aktion} Der \emph{Benutzer*} klickt die Schaltfläche "`neue Zeiterfassung".' \\
            \textbf{Reaktion} Es werden die Elemente für eine neue \emph{Zeiterfassung} geladen.
            \item
                \textbf{Stand} Der \emph{Benutzer*} sieht die Elemente für eine neue \emph{Zeiterfassung}. \\
                \textbf{Aktion} Der \emph{Benutzer*}
                    \begin{itemize}
                        \item weist der \emph{Zeiterfassung} eine \emph{Kategorie} über das Textfeld "`Kategorie"' zu.
                        \item weist der \emph{Zeiterfassung} eine \emph{Tätigkeit} über das Textfeld "`Tätigkeit"' zu.
                        \item startet die \emph{Zeiterfassung} über die Schaltfläche "`Start"'.
                    \end{itemize}
                \textbf{Reaktion} Die \emph{Zeiterfassung} läuft.
        \req{T122}
            \textbf{Stand} Die \emph{Zeiterfassung} läuft. \\
            \textbf{Aktion} Der \emph{Benutzer*} stoppt die \emph{Zeiterfassung} über die Schaltfläche "`Zeiterfassung stoppen"'. \\
            \textbf{Reaktion}
                \begin{itemize}
                    \item Die \emph{Zeiterfassung} stoppt.
                    \item Die erfasste Zeit wird in die Liste der erfassten Zeiten eingetragen.
                \end{itemize}
	\end{requirements}


	\req{T130} \emph{Benutzer*} erhält \emph{Warnung} wegen nicht abgegebenem \emph{Studenzettel}. \\
        \textbf{Stand} Der \emph{Benutzer*} ist nicht eingeloggt. \\
        \textbf{Aktion} Der \emph{Benutzer*} meldet sich an \emph{T110}. \\
        \textbf{Reaktion} Der \emph{Benutzer*} erhält nach der Anmeldung ein Popup, dass ihn auffordert, den \emph{Stundenzettel} abzugeben.

	\req{T140} \emph{Betreuer*} prüft neu \emph{abgebenen Stundenzettel}.
	\begin{requirements}
	    \req{T141}
	        \textbf{Stand} Der \emph{Betreuer*}
	            \begin{itemize}
	                \item ist angemeldet.
	                \item erhält über die Benachrichtigungs-Schaltfläche mittgeteilt das ein neuer \emph{Stundenzettel} abegeben wurde.
	            \end{itemize}
            \textbf{Aktion} Der \emph{Betreuer*} klickt auf die Benachrichtigung. \\
            \textbf{Reaktion} Die entsprechende \emph{Stundenzettelseite} wird  geladen.
	    \req{T142}
	        \textbf{Stand} Die \emph{Stundenzettelseite} ist geladen. \\
            \textbf{Aktion}
                \begin{itemize}
                    \item Der \emph{Betreuer*} überprüft den \emph{Stundenzettel}.
                    \item \textit{Stundenzettel Korrekt} Der \emph{Betreuer*} klickt die Schaltfläche Stundenzettel geprüft.
                    \item \textit{Stundenzettel nicht Korrekt}  Der \emph{Betreuer*} klickt die Schaltfläche Stundenzettel nicht Okay und kommentiert, was nicht korrekt ist.
                \end{itemize}
            \textbf{Reaktion}
                \begin{itemize}
                    \item Die \emph{Stundenzettelseite} wird geschlossen der \emph{Betreuer*} wird zur \emph{Hauptseite} zurückgeleitet.
                    \item \textit{Stundenzettel Korrekt} Der \emph{Stundenzettel} ist als \emph{geprüft} markiert und der \emph{Administrator*} wird darüber benachrichtigt.
                    \item \textit{Stundenzettel nicht Korrekt} Der \emph{Stundenzettel} wird als \emph{nicht abgegeben} markiert und der \emph{Benutzer*} darüber benachrichtigt.
                \end{itemize}
	\end{requirements}

	\req{T150} \emph{Benutzer*} editiert die \emph{Tätigkeit} in einem \emph{nicht abgegebenem} \emph{Stundenzettel}.
	\begin{requirements}
	    \req{T151}
	        \textbf{Stand} \emph{Benutzer*} befindet sich auf seiner \emph{Hauptseite} mit den Auflistungen der eingetragenen Zeiten. \\
	        \textbf{Aktion} Der \emph{Benutzer*} klickt dem Editieren Button bei einer eingetragenen Zeit.\\
            \textbf{Reaktion} Die Elemente der eingetragenen Zeit werden editierbar.
        \req{T152}
        \textbf{Stand} Die Elemente der eingetragenen Zeit sind editierbar. \\
        \textbf{Aktion} Der \emph{Benutzer*}
            \begin{itemize}
                \item klickt in das Textfeld in dem die \emph{Tätigkeit} vermerkt wird.
                \item ändert die dort eingetragene \emph{Tätigkeit}.
                \item klickt auf die Speichern Schaltfläche.
            \end{itemize}
            \textbf{Reaktion} Die eingetragene \emph{Tätigkeit} ist damit geändert.
    \end{requirements}

    \req{T160} Der \emph{Benutzer*} wählt die Seite an und besitz ein gültiges Session Cookie. \\
        \textbf{Stand} Offenes Browserfenster. \\
        \textbf{Aktion} Der \emph{Benutzer*} gibt die lokae URL der ChronoCommand-Instanz in die Browserzeile ein in betätigt Enter. \\
        \textbf{Reaktion} Da Session Cookie geladen, wird die \emph{Benutzer* Hauptseite} gezeigt.

    \req{T170} Der \emph{Benutzer*} verletzt gesetzliche Vorgaben. \\
        \textbf{Stand} \emph{Zeiterfassung} läuft. \\
        \textbf{Aktion} Die \emph{Zeiterfassung} überschreitet ein gesetzes Limit. \\
        \textbf{Reaktion}
            \begin{itemize}
                \item Der \emph{Benutzer*} erhält eine \emph{Warnung}, dass er gesetzliche Vorgaben verletzt.
                \item Die \emph{Zeiterfassung} stoppt.
            \end{itemize}

    \req{T180} Der \emph{Betreuer*} sieht \emph{Stundendaten} ein. \\
        \textbf{Stand} Der \emph{Betreuer*} befindet sich auf der \emph{Betreuer* Hauptseite}. \\
        \textbf{Aktion} Der \emph{Betreuer*} klickt die Schaltfläche "`Stundenzettel einsehen"'. \\
        \textbf{Reaktion} Der \emph{Betreuer*} erhält die \emph{Stundendaten} der ihm zugewiesenen \emph{Benutzer*}.
        
    \req{T190} Der \emph{Benutzer*} trägt manuell eine Zeit ein.
    \begin{requirements}
        \req{T191}
            \textbf{Stand} Der \emph{Benutzer*} befindet sich angemeldet auf der \emph{Hauptseite}. \\
            \textbf{Aktion} Der \emph{Benutzer*} klickt auf die Schaltfläche "`Zeit manuell eintragen"'. \\
            \textbf{Reaktion} Die Seite zur manuellen Eintragung von Zeiten wird geladen.
        \req{T192}
            \textbf{Stand} Die Seite zur manuellen Eintragung von Zeiten ist geladen. \\
            \textbf{Aktion} 
                \begin{itemize}
                    \item Der \emph{Benutzer*} gibt die Daten für eine vollständige \emph{Zeiterfassung} ein.
                    \item Der \emph{Benutzer*} klickt auf die Schaltfäche "`Speichern"'.
               \end{itemize}
            \textbf{Reaktion} Die Daten werden als erfasste Zeit in die Liste der erfassten Zeiten eingetragen.
        \req{193}
            \textbf{Stand} Der \emph{Benutzer*} befindet sich angemeldet auf der \emph{Hauptseite}. \\
            \textbf{Aktion}
                \begin{itemize}
                    \item Der \emph{Benutzer*} gibt unvollständige Daten für eine \emph{Zeiterfassung} ein.
                    \item Der \emph{Benutzer*} klickt auf die Schaltfäche "`Speichern"'.
                \end{itemize}
            \textbf{Reaktion} Der \emph{Benutzer*} erhält eine Warnung, dass die Daten unvollständig sind.
    \end{requirements}

    \req{T200} Der \emph{Benutzer*} löscht eine erfasste Zeit.
    \begin{requirements}
        \req{T201}
            \textbf{Stand} Der \emph{Benutzer*} befindet sich angemeldet auf seiner \emph{Hauptseite} mit den Auflistungen der eingetragenen Zeiten. \\
            \textbf{Aktion}
                \begin{itemize}
                    \item Der \emph{Benutzer*} wählt eine eingetragene Zeit aus der Liste.
                    \item Der \emph{Benutzer*} klickt auf die Schaltfläche "`Löschen".
                \end{itemize}
            \textbf{Reaktion} Die eingetragene Zeit wird gelöscht.
    \end{requirements}
    
\newpage
    \req{T210} Der \emph{Benutzer*} stoppt eine unvollständige \emph{Zeiterfassung}.
    \begin{requirements}
        \req {211}
            \textbf{Stand}
            \begin{itemize}
                \item Die \emph{Zeiterfassung} läuft.
                \item Der \emph{Benutzer*} hat keine \emph{Tätigkeit} eingetragen.
            \end{itemize}
            \textbf{Aktion} Der \emph{Benutzer*} stoppt die \emph{Zeiterfassung}. \\
            \textbf{Reaktion}
            \begin{itemize}
                \item Die \emph{Zeiterfassung} läuft weiter.
                \item Der \emph{Benutzer*} erhält eine Warnung über die nicht eingetragene \emph{Tätigkeit}.
            \end{itemize}
        \req{212}
            \textbf{Stand}
            \begin{itemize}
                \item Die \emph{Zeiterfassung} läuft.
                \item Der \emph{Benutzer*} hat keine \emph{Kategorie} eingetragen.
            \end{itemize}
            \textbf{Aktion} Der \emph{Benutzer*} stoppt die \emph{Zeiterfassung}. \\
            \textbf{Reaktion}
            \begin{itemize}
                \item Die \emph{Zeiterfassung} läuft weiter.
                \item Der \emph{Benutzer*} erhält eine Warnung über die nicht eingetragene \emph{Kategorie}.
            \end{itemize}
    \end{requirements}
    
    \req{T220} Der \emph{Administrator*} schaut sich alle Daten an.
    \begin{requirements}
        \req{221}
            \textbf{Stand} Der \emph{Administrator*} befindet sich angemeldet auf der \emph{Hauptseite}. \\
            \textbf{Aktion} Der \emph{Administrator*} navigiert zur \emph{Übersicht aller Daten}. \\
            \textbf{Reaktion} Der \emph{Administrator*} erhält eine graphische Übersicht aller erfasster Zeiten.
        \req{222}
            \textbf{Stand} Der \emph{Administrator*} befindet sich auf der Seite \emph{Übersicht aller Daten}. \\
            \textbf{Aktion} Der \emph{Administrator*} wählt eine bestimmte \emph{Kategorie} aus. \\
            \textbf{Reaktion} Die Übersicht zeigt alle geleisteten Stunden dieser \emph{Kategorie}.
    \end{requirements}
            
    
\end{requirements}
