\section{Globale Testfälle}

\subsection{Funktionssequenzen}
\begin{requirements}
	\req{T110} \em Benutzer \em Meldet sich an
	\begin{requirements}
  			\req{T111}
  			    \textbf{Stand} Offenes Browserfenster. \\
  			    \textbf{Aktion} Der \em Benutzer \em gibt chronoCommand.eu in die Browserzeile ein und bestätigt mit Enter. \\
  			    \textbf{Reaktion} Da kein Session Cookie geladen ist, wird der Anmeldescreen gezeigt.
  			\req{T112}
  			    \textbf{Stand} Die Anmeldeseite wurde geladen.\\
                \textbf{Aktion} Der \em Benutzer \em
                    \begin{itemize}
                        \item gibt im Nutzernamen Feld seinen Nutzernamen an.
                        \item gibt im Passwort Feld sein Passwort an.
                        \item betätigt die Anmelden Schaltfläche.
                    \end{itemize}
                \begin{itemize}
			\item \textbf{Reaktion} \em Benutzerdaten korrekt \em Der \em Benutzer \em wird angemeldet.
			\item \textbf{Reaktion} \em Benutzerdaten nicht korrekt \em Der \em Benutzer \em bekommt eine Fehlermeldung.
                \end{itemize}

	\end{requirements}
	\req{T120} Angemeldeter \em Benutzer \em startet/stoppt eine \em Zeiterfassung \em.
	\begin{requirements}
	    \req{T121}
            \textbf{Stand} Der \em Benutzer \em befindet sich angemeldet auf der \em Hauptseite \em. \\
            \textbf{Aktion} Der \em Benutzer \em klickt die Schaltfläche "`neue Zeiterfassung"' \\
            \textbf{Reaktion} Es werden die Elemente für eine neue \em Zeiterfassung \em geladen.
            \item
                \textbf{Stand} Der \em Benutzer \em sieht die Elemente für eine neue \em Zeiterfassung \em. \\
                \textbf{Aktion} Der \em Benutzer \em
                    \begin{itemize}
                        \item weist der \em Zeiterfassung \em eine \em Kategorie \em über das Textfeld "`Kategorie"' zu.
                        \item weist der \em Zeiterfassung \em eine \em Tätigkeit \em über das Textfeld "`Tätigkeit"' zu
                        \item startet die \em Zeiterfassung \em über die Schaltfläche "`Start"'
                    \end{itemize}
                \textbf{Reaktion} Die \em Zeiterfassung \em läuft.
        \req{T122}
            \textbf{Stand} Die \em Zeiterfassung \em läuft. \\
            \textbf{Aktion} Der \em Benutzer \em stoppt die \em Zeiterfassung \em über die Schaltfläche "`Zeiterfassung stoppen"' \\
            \textbf{Reaktion}
                \begin{itemize}
                    \item Die \em Zeiterfassung \em stoppt.
                    \item Die Erfasste Zeit wird in die erfassten Zeiten eingetragen.
                \end{itemize}
	\end{requirements}


	\req{T130} \em Benutzer \em erhält \em Warnung \em wegen nicht abgegebenem \em Studenzettel \em \\
        \textbf{Stand} Der \em Benutzer \em ist nicht eingeloggt. \\
        \textbf{Aktion} Der \em Benutzer \em meldet sich an \em T110 \em. \\
        \textbf{Reaktion} Der \em Benutzer \em erhält nach der Anmeldung ein Popup, dass ihn auffordert, den \em Stundenzettel \em abzugeben.

	\req{T140} \em Betreuer \em prüft neu \em abgebenen Stundenzettel \em
	\begin{requirements}
	    \req{T141}
	        \textbf{Stand} Der \em Betreuer \em
	            \begin{itemize}
	                \item ist angemeldet.
	                \item erhält über die Benachrichtigungs-Schaltfläche mittgeteilt das ein neuer \em Stundenzettel \em abegeben wurde.
	            \end{itemize}
            \textbf{Aktion} Der \em Betreuer \em klickt auf die Benachrichtigung. \\
            \textbf{Reaktion} Die entsprechende \em Stundenzettelseite \em wird  geladen.
	    \req{T142}
	        \textbf{Stand} Die \em Stundenzettelseite \em ist geladen. \\
            \textbf{Aktion}
                \begin{itemize}
                    \item Der \em Betreuer \em überprüft den \em Stundenzettel \em.
                    \item \textit{Stundenzettel Korrekt} Der \em Betreuer \em klickt die Schaltfläche Stundenzettel geprüft.
                    \item \textit{Stundenzettel nicht Korrekt}  Der \em Betreuer \em klickt die Schaltfläche Stundenzettel nicht Okay und kommentiert, was nicht korrekt ist.
                \end{itemize}
            \textbf{Reaktion}
                \begin{itemize}
                    \item Die \em Stundenzettelseite \em wird geschlossen der \em Betreuer \em wird zur \em Hauptseite \em zurückgeleitet
                    \item \textit{Stundenzettel Korrekt} Der \em Stundenzettel \em ist als \em geprüft \em markiert und der \em Administrator \em wird darüber benachrichtigt
                    \item \textit{Stundenzettel nicht Korrekt} Der \em Stundenzettel \em wird als \em nicht abgegben \em markiert und der \em Benutzer \em darüber benachrichtigt
                \end{itemize}
	\end{requirements}

	\req{T150} \em Benutzer \em editiert die \em Tätigkeit \em in einem \em nicht abgegebenem \em \em Stundenzettel \em
	\begin{requirements}
	    \req{T151}
	        \textbf{Stand} \em Benutzer \em befindet sich auf seiner \em Hauptseite \em mit den Auflistungen der eingetragenen Zeiten. \\
	        \textbf{Aktion} Der \em Benutzer \em klickt dem Editieren Button bei einer eingetragenen Zeit.\\
            \textbf{Reaktion} Die Elemente der eingetragenen Zeit werden editierbar.
        \req{T152}
        \textbf{Stand} Die Elemente der eingetragenen Zeit sind editierbar. \\
        \textbf{Aktion} Der \em Benutzer \em
            \begin{itemize}
                \item klickt in das Textfeld in dem die \em Tätigkeit \em vermerkt wird.
                \item ändert die dort eingetragene \em Tätigkeit \em.
                \item klickt auf die Speichern Schaltfläche.
            \end{itemize}
            \textbf{Reaktion} Die eingetragene \em Tätigkeit \em ist damit geändert
    \end{requirements}

    \req{T160} Der \em Benutzer \em wählt die Seite an und besitz ein gültiges Session Cookie. \\
        \textbf{Stand} Offenes Browserfenster. \\
        \textbf{Aktion} Der \em Benutzer \em gibt chronoCommand.eu in die Browserzeile ein in betätigt Enter. \\
        \textbf{Reaktion} Da Session Cookie geladen, wird die \em Benutzer Hauptseite \em gezeigt.

    \req{T170} Der \em Benutzer \em verletzt gesetzliche Vorgaben. \\
        \textbf{Stand} \em Zeiterfassung \em läuft. \\
        \textbf{Aktion} Die \em Zeiterfassung \em überschreitet ein gesetzes Limit. \\
        \textbf{Reaktion}
            \begin{itemize}
                \item Der \em Benutzer \em erhält eine \em Warnung \em, dass er gesetzliche Vorgaben verletzt.
                \item Die \em Zeiterfassung \em stoppt.
            \end{itemize}

    \req{T180} \em Betreuer \em sieht \em Stundendaten \em ein. \\
        \textbf{Stand} Der \em Betreuer \em befindet sich auf der \em Betreuer Hauptseite \em. \\
        \textbf{Aktion} Der \em Betreuer \em klickt die Schaltfläche "`Stundenzettel einsehen"' \\
        \textbf{Reaktion} Der \em Betreuer \em erhält die \em Stundendaten \em der ihm zugewiesenen \em Benutzer \em.

\end{requirements}
