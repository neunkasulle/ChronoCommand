\section{Einleitung}

Auf herkömmlicher Weise muss ein \emph{Benutzer*} (Studierende) seinen \emph{Stundenzettel} monatlich manuell erstellen und an seinen \emph{Betreuer*} übergeben. Nach einer erfolgreichen Kontrolle muss der \emph{Betreuer*} den \emph{Stundenzettel} an das Sekretariat zur Bearbeitung weiterleiten. \\

Es können an einer Universität hunderte bis tausende Studierende beschäftigt sein. Diese manuellen Vorgänge sind extrem zeitaufwändig und außerdem könnten viele Fehler sowie Probleme dabei auftreten.\\

Das Ziel des Projekts ist, ein professionelles und komfortables Zeiterfassungssystem als Weboberfläche zu entwickeln, mit dem die \emph{Stundenzettel} von Angestellten an Universitäten digital statt manuell erfasst werden können und die Verwaltungsarbeiten von \emph{Betreuern*} und \emph{Administratoren*} erleichtert werden können.\\

Mit diesem digitalen Zeiterfassungssystem sollen folgende Vorteile erreicht werden:

\begin{itemize}
	\item Durch die automatische Datenübernahme wird der manuelle Aufwand auf ein Minimum reduziert, das heißt Zeit- und Kostenersparnis.
	\item Das Verletzen von gesetzlichen Vorgaben wird verhindert, z.B. Arbeiten an Feiertagen oder über dem gesetzlichen Maximum.
	\item Eine effiziente Projektverwaltung durch die getrennte Erfassung verschiedener Tätigkeiten.
	\item Eine übersichtlichere Verwaltung durch grafische und tabellarische Darstellung.
\end{itemize}
