\section{Einleitung}

Auf herkömmlicher Weise muss ein Benutzer (HiWi) seinen Stundenzettel monatlich manuell erstellen und an seinen Betreuer übergeben. Nach einer erfolgreichen Kontrolle muss der Betreuer den Stundenzettel an das Sekretariat zur Bearbeitung weiterleiten. \\

Es können an einer Universität hunderte bis tausende HiWis beschäftigt sein. Diese manuellen Vorgänge sind extrem zeitaufwändig und außerdem könnten viele Fehler sowie Probleme dabei auftreten.\\

Das Ziel des Projekts ist, ein professionelles und komfortables Zeiterfassungssystem als Web Front-end zu entwickeln, mit dem die Stundenzettel von HiWis an Universitäten digital statt manuell erfasst werden können und die Verwaltungsarbeiten von Betreuern und Admin erleichtern werden können.\\

Mit diesem digitalen Zeiterfassungssystem sollen folgende Vorteile erreicht werden:\\

\begin{itemize}
	\item Durch die automatische Datenübernahme wird der manuelle Aufwand auf ein Minimum reduziert, das heißt Zeit- und Kostenersparnis.
	\item verhindert Verletzungen der Vorgabe des Gesetzes, z.B. Arbeiten an Feiertagen, Arbeiten über Gesetzlimit usw.
	\item effiziente Projektverwaltung durch die getrennte Erfassung verschiedener Tätigkeiten.
	\item übersichtlichere Verwaltung durch grafische und tabellarische Darstellung

\end{itemize}
