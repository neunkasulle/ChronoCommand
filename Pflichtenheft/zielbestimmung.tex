\section{Zielbestimmung}

\subsection{Musskriterien}

\begin{itemize}
	\item Starten und Stoppen von einer Zeiterfassung
	\item Zeiten sind nachträglich erfassbar
	\item Zeiten sind nachträglich änderbar
	\item erfasste Zeiten können Tätigkeiten zugeordnet werden
	\item erfasste Zeiten können Kategorien zugeordnet werden
	\item Warnungen wenn Zeiten nicht eingetragen sind
	\item Möglichkeit die erfassten Zeiten an den Admin zu übersenden
	\item Warnungen wenn gesetzlich Pausen genommen werden müssen
	\item Warnungen wenn zu erfasster Zeit keine Tätigkeit zugeordnet ist
	\item Warnungen wenn zu erfasster Zeit keine Kategorie zugeordnet ist
	\item Vergangene Zeiten sind einsehbar
	\item Hinzufügen von Accounts
	\item Löschen von Accounts
	\item Ändern von Accounts
	\item Bestimmte Benutzer sind Administratoren
	\item Bestimmte Benutzer sind Betreuer
	\item Administratoren legen fest welche Benutzer von welchem Betreuer betreut werden
	\item Backups werden regelmäßig angefertigt
	\item Es kann zwischen LDAP und lokalen Accounts zur Benutzerverwaltung gewählt werden
	\item Zeiten sollen durch eine graphische Übersicht visulisiert werden (Heatmap, Punch Card) TODO überprüfen
\end{itemize}


\subsection{Wunschkriterien}

\begin{itemize}
	\item Mehrere Frontendimplementierungen sind möglich
	\item Zeitvorhersagen anhand vergangener Arbeitszeit
	\item Überwachung der Arbeitszeit auch ohne abgebene Zeiten möglich
	\item Erfassung der Tätigkeiten während der Arbeitszeit
	\item Möglichkeit die IP-Range, von der aus sich Administrator einloggen können, beschränken
	\item Kalendarische Übersicht an erfasster Zeit
	\item Zeiten sollen durch eine graphische Übersicht visulisiert werden TODO auf aktualität überprüfen
	\item Über eine Toolbar Zeit erfassen und eine Tätigkeit zuordnen
\end{itemize}


\subsection{Abgrenzungskriterien}
\begin{itemize}
	\item Korrekter Ablauf im Internet Explorer wird nicht unterstüzt
	\item Die Nutzung der gleichen Instanz durch mehrere Firmen und die damit erforderliche Trennung der Daten wird nicht unterstützt
	\item Es wird nur auf Einhaltung der deutschen Gesetzgebung in Baden-Württemberg geachtet
\end{itemize}
