\section{Produkteinsatz}
Das System dient zur Verwaltung und Kontrolle der Arbeitszeiten von Mitarbeitern* (Studierende), Analyse verschiedener Projekte anhand der erfassten \emph{Zeiten} und der entsprechenden \emph{Tätigkeiten}.
Die von \emph{Benutzern*} angegebenen Daten werden auf einem Server gespeichert. Auf dem Server gespeicherte Daten können mit einem Internet Browser aufgerufen, tabellarisch und grafisch dargestellt werden.
Durch Nachrichtenfunktionen soll der Kontakt zwischen \emph{Benutzern*}, \emph{Betreuern*} und \emph{Administratoren*} sichergestellt werden.
Des Weiteren gibt es eine Benutzerverwaltung, um die Zugriffsrechte auf die Daten zu verwalten.
\subsection{Anwendungsbereiche}
\begin{itemize}
	\item Der Anwendungsbereich umfasst sämtliche gewerbliche Betriebsumfelder sowie Universitäten, Institutionen, Vereine,
	welche die Arbeitszeiten von Studierenden oder ähnlichen Mitarbeitern* auf effizienter Weise verwalten wollen. Die Anzahl der Studierenden bzw. Mitarbeitern* sind normalerweise groß.
\end{itemize}

\subsection{Zielgruppen}
\begin{itemize}
	\item Die Anwender in den Zielgruppen sind \emph{Benutzer*}, \emph{Betreuer*} und \emph{Administratoren}*.
	Die Zielgruppen haben ein gruppeneigenes Berechtigungsprofil, das auf die Benutzerbedürfnisse zugeschnitten ist und nur Zugriff auf die nötigen Funktionen erlaubt.
	Dabei hat der \emph{Benutzer*} nur Zugriffsberechtigung auf seine eigenen Daten, während der \emph{Betreuer*} auf alle seinen zugewiesenen \emph{Benutzer*} und der \emph{Administrator*} auf alle Daten Zugriff hat.
\end{itemize}

\subsection{Betriebsbedingungen}
\begin{itemize}
	\item Die Betriebsbedingungen müssen für die Anwendung auf einem zentralen Webserver spezifiziert werden. Der Server läuft im Dauerbetrieb und unbeaufsichtigt.
	Client-User brauchen einen normalen Internetfähigen Rechner oder Smartphone.
\end{itemize}
