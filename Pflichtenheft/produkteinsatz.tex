\section{Produkteinsatz}
Das System dient zur Verwaltung und Kontrolle der Arbeitszeiten von Mitarbeiter (HIWIs), Analyse verschiedener Projekte anhand der erfassten Zeiten und der entsprechenden Tätigkeiten. Die von Benutzern angegebenen Daten werden auf einem Server gespeichert. Auf dem Server gespeicherte Daten können mit einem Internet Browser aufgerufen, tabellarisch und grafisch dargestellt werden. Durch Nachrichtenfunktionen soll der Kontakt zwischen HIWIs, Betreuern und Admin sichergestellt werden. Des Weiteren gibt es eine Benutzerverwaltung, um die Zugriffsrechte auf die Daten zu verwalten.
\subsection{Anwendungsbereiche}
\begin{itemize}
	\item Der Anwendungsbereich umfasst sämtliche gewerbliche Betriebsumfelder sowie Universitäten, Institutionen, Vereine, welche die Arbeitszeiten von HIWIs oder ähnlichen Mitarbeitern auf effizienter Weise verwalten wollen. Die Anzahl der HIWIs bzw. Mitarbeitern sind normalerweise groß. 
\end{itemize}

\subsection{Zielgruppen}
\begin{itemize}
	\item Die Anwender in den Zielgruppen sind HIWI, Betreuer und Administrator. Die Zielgruppen haben ein gruppeneigenes Berechtigungsprofil, das auf die Benutzerbedürfnisse zugeschnitten ist und nur Zugriff auf die nötigen Funktionen erlaubt. Dabei hat HIWI nur Zugriffsberechtigung auf seine eigenen Daten, während der Betreuer auf alle seinen zugewiesenen HIWIs und der Administrator auf alle Daten Zugriff hat.
\end{itemize}

\subsection{Betriebsbedingungen}
\begin{itemize}
	\item Die Betriebsbedingungen müssen für die Anwendung auf einem zentralen Webserver spezifiziert werden. Der Server läuft im Dauerbetrieb und unbeaufsichtigt. Client-User brauchen einen normalen Internetfähigen Rechner oder Smartphone.
\end{itemize}
