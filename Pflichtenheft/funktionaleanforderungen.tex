\section{Funktionale Anforderungen}

Die den Wunschkriterien zugeordneten Anforderungen sind mit einem "`+"' hinter der Indentifikationsnummer markiert.

\subsection{Stundenzettel und Zeiterfassung}

\begin{requirements}
    \req[Zeiterfassung]{F110}
    Ein User kann in seiner Hauptseite eine \em{Zeiterfassung} starten und stoppen.
    \begin{requirements}
        \req[Kategorie]{F111} Eine \em{Zeiterfassung} ist mit einer \em{Kategorie} versehen.
        \req[Tätigkeit]{F112} Eine \em{Zeiterfassung} ist mit einer \em{Tätigkeit} verbunden.
        \req[Zeiterfassung ändern]{F113} Eine \em{Zeiterfassung} kann nachträglich geändert werden.
        \req[Alternative Zeiterfassung]{F114} Eine \em{Zeiterfassung} kann auch manuell ohne das Starten und Stoppen einer Zeit erfasst werden.
        \req[Zeiterfassung löschen]{F115} Eine erfasste Zeit kann vom \em{Benutzer} wieder gelöscht werden.
    \end{requirements}

    \req[Gesetzliche Vorgaben]{F120}
    Die \em{Zeiterfassung} und der \em{Stundenzettel} folgen den gesetzlichen Bestimmungen.
    \begin{requirements}
        \req[Maximale Arbeitszeit]{F121} Die gesetzlich maximale Arbeitszeit kann bei einer \em{Zeiterfassung} nicht überschritten werden \em{(siehe §3 ArbZG)}.
        \req[Pausenzeiten]{F122} Eine \em{Zeiterfassung} kann nicht dem \em{Stundenzettel} hinzugefügt werden wenn für ihren Umfang gesetzliche Pausenzeiten nicht eingetragen wurden (siehe §4 ArbZG).
        \req[Ruhezeiten]{F123} Die Einhaltung der Ruhezeiten wird erzwungen (siehe §5 ArbZG).
        \req[Nachtarbeit]{F124} \em{Administratoren} können Nachtarbeit für einzelne Nutzer erlauben, allen anderen Nutzern ist Nachtarbeit nicht gestattet (siehe §6 ArbZG).
        \req[Sonn- und Feiertage]{F125} Die Einhaltung der Sonn- und Feiertagsruhe wird erzwungen (siehe §9 ArbZG).
		Dabei werden nur die Feiertage in Baden-Württemberg beachtet.
                Die Sonn- und Feiertagsruhe kann für einzelne \em{Benutzer} von \em{Administratoren} deaktiviert werden (siehe §10 ArbZG), dabei wird dann §11 ArbZG beachtet.
        \req[Manipulation verhindern]{F126} Die gesetzlichen Vorgaben sind über eine Konfigurationsdatei anpassbar deren Prüfsumme im Code hinterlegt ist \em{(auch F410)}.
                Daher muss zum Ändern der gesetzlichen Vorgaben das Programm neu compiliert werden.
    \end{requirements}

    \req[Stundenzettel]{F130}
    Der \em{Stundenzettel} stellt eine Aufzeichnung der Arbeitstunden in einem vorgesetzten Zeitraum dar.
    \begin{requirements}
        \req[Sichtbarkeit]{F131} Der \em{Betreuer} und der \em{Administrator} können den Stundenzettel einsehen.
        \req[Stundenzettel abgeben]{F132} Ist der \em{Benutzer} mit seiner \em{Zeiterfassung} in einem Zeitraum fertig, so kann er den \em{Stundenzettel abgeben}.
        \req[abgegebener Stundenzettel]{F133} Ein \em{abgegebener Stundenzettel} kann vom \em{Benutzer} nichtmehr verändert werden.
        \req[abgegebene Stundenzettel]{F134} Der \em{Betreuer} und der \em{Administrator} werden über \em{abgegebene Stundenzettel} informiert
        \req[Stundenkonto]{F135} Im \em{Stundenzettel} ist die tatsächliche und die zu leistende Arbeitszeit sichtbar.
        \req[Betreuerkontrolle]{F136} Ein \em{Stundenzettel} kann vom \em{Betreuer} als \em{geprüft} eingestuft werden werden.
        \req[Betreuerkontrolle erfolgreich]{F137} Ein vom \em{Betreuer} als \em{geprüft} eingestufter \em{Stundenzettel} wird dem \em{Administrator} als \em{abgegeben} angezeigt.
        \req[Betreuerkontrolle fehlgeschlagen]{F138} Ein vom \em{Betreuer} abgelehnter \em{Stundenzettel} wird als \em{nicht abgegeben} markiert, vom \em{Betreuer} kommentiert, und der \em{Benutzer} darüber informiert.
        \req[Stundenzettel exportieren]{F139} Ein \em{Administrator} kann alle \em{abgegebenen Stundenzettel} eines Monats zum Drucken exportieren.
        \req[Drucken]{F13A}\em{Benutzer}, \em{Betreuer} und \em{Administratoren} können \em{Stundenzettel} drucken.
    \end{requirements}

\end{requirements}

\subsection{Benutzer und Rechte}

\begin{requirements}
    \req[Benutzer]{F210}
    Ein \em{Benutzer} stellt den Standart User dar.
    \begin{requirements}
        \req[Stundenzettel]{F211} Ein \em{Benutzer} kann nur seinen eigenen \em{Stundenzettel} einsehen und verändern.
        \req[Warnungen]{F212} Ein \em{Benutzer} erhält nur \em{Warnungen} über seine eigenen Aktionen.
        \req[Erinnerungen]{F213} Ein \em{Benutzer} erhält nur \em{Erinnerungen}, die ihn selbst betreffen.
    \end{requirements}

    \req[Betreuer]{F220}
        Ein \em{Betreuer} ist für mehrere \em{Benutzer} zuständig.
        \begin{requirements}
            \req[Benutzer zuweisen]{F221} Einem \em{Betreuer} können \em{Benutzer} zugewiesen werden
            \req[Stundenzettel]{F222} Der \em{Betreuer} kann über seine Übersicht alle ihm zugewiesenen \em{Benutzer} einsehen
            \req[Warnungen]{F223} Der \em{Betreuer} erhält \em{Warnungen für alle \em{Benutzer} die ihn zugewiesen sind.
            \req[Erinnerungen]{F224} Der \em{Betreuer} erhält \em{Erinnerungen} für alle \em{Benutzer} die ihm zugewiesen sind.
            \req[Stundenzettel prüfen]{F225} Der \em{Betreuer} überprüft die \em{abgegebenen Stundenzettel} der im zugewiesenen \em{Benutzer}.
        \end{requirements}

    \req[Administrator]{F230}
        Ein \em{Admin} stellt das Verwaltungsorgan dar.
        \begin{requirements}
            \req[Benutzer erstellen]{F231} Der \em{Administrator} kann neue \em{Benutzer} anlegen.
            \req[Benutzer editieren]{F232} Der \em{Administrator} kann die Daten existierender \em{Benutzer} verändern.
            \req[Benutzer löschen]{F233} Der \em{Administrator} kann einen existierenden \em{Benutzer} löschen.
            \req[Benutzer zuweisen]{F234} Der \em{Administrator} kann \em{Benutzer} einem \em{Betreuer} zuweisen.
            \req[Betreuer Ansicht]{F235} Der \em{Administrator} kann für jedes \em{Team} auch auf die \em{Betreueransicht} zugreifen.
        \end{requirements}
\end{requirements}

\subsection{Zeitüberwachung und Darstellung}
    \begin{requirements}
        \req[Graphische Darstellung Zeit]{F310+}
        \begin{requirements}
            \req[Übersicht Teams]{F311} Der \em{Administrator} soll die bisher benötigte Zeit aller \em{Teams} auf seiner \em{Hauptseite} einsehen können.
            \req[Übersicht Betreuer]{F312} Dem \em{Betreuer} soll die bisher benötigte Zeit aller im zugewiesenen \em{Benutzer} angezeigt werden.
            \req[Übersicht Benutzer]{F313} Der \em{Benutzer} soll seine bisher aufgewedete Zeit graphisch angezeigt werden.
            \req[Burn Rate Benutzer]{F314} Der \em{Benutzer} soll seine durchschnittliche wöchentliche Stundenrate(Burn Rate) angezeigt bekommen.
        \end{requirements}

        \req[Graphische Darstellung Stundenzettelabgabe]{F320}
        \begin{requirements}
            \req[Übersicht Teams]{F321} Der \em{Administrator} soll auf seiner \em{Hauptseite} die bisherigen \em{Abgaben} von \em{Stundenzetteln} angezeigt bekommen.
            \req[Übersicht Betreuer]{F322} Der \em{Betreuer} soll den bisherigen Abgabefortschritt seines \em{Teams} dargestellt bekommen.
        \end{requirements}

        \req[Darstellung von Daten]{F330}
        \begin{requirements}
            \req[Heatmap]{F331} \em{Benutzer} können einsehen an welchen Tagen die meiste Arbeitszeit geloggt wurde.
            \req[Punch Card]{F332} \em{Benutzer} können einsehen zu welchen Zeiten die meiste Arbeitszeit geloggt wurde.
        \end{requirements}

        \req[Tätigkeiten Darstellung]{F340}
        \begin{requirements}
            \req[Tätigkeiten Ranking]{F341} \em{Benutzer} können sich den Zeitverbrauch pro Tätigkeit, gesammelt über alle \em{Benutzer}, anzeigen lassen.
            \req[Tätigkeits Heatmap]{F342} \em{Benutzer} können sich eine Tätigkeits \em{Heatmap}, anzeigen lassen.
        \end{requirements}
        
        \req[Übersicht aller Daten]{F350}
        \begin{requirements}
             \req[Admin]{F351} \em{Administratoren} können alle geleisteten Stunden aller \em{Benutzer} graphisch einsehen.
             \req[Filter]{F352} Die graphische Anzeige erlaubt es nach \em{Kategorien}, \em{Betreuern}, \em{Benutzer} und Datum zu filtrieren.
        \end{requirements}

       \req[Framework zur Darstellung]{F360} Daten werden durch ein Statistik Framework visualisiert angezeigt.
    \end{requirements}

\subsection{Sonstige FA}
    \begin{requirements}
        \req[Sicherheit von Konfigurationen]{F410} Konfigurationsdateien sind durch einen Hash vor dem Editieren geschützt
        \req[Sicherheit der Datenbank]{F420} Die Datenbank wird durch einen Hash vor externem Editieren geschützt
        \begin{requirements}
            \req[Externe Veränderung der Datenbank]{F421} Sollte eine Veränderung stattgefunden haben, werden alle Nutzer über dieses externe Event informiert.
        \end{requirements}
    \end{requirements}

