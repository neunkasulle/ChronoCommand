\section{Funktionale Anforderungen}

Die den Wunschkriterien zugeordneten Anforderungen sind mit einem "`+'" hinter der Indentifikationsnummer markiert.

\subsection{Stundenzettel und Zeiterfassung}

\begin{requirements}
    \req[Zeiterfassung]{F110}
    Ein User kann in seiner Hauptseite eine Zeiterfassung starten und stoppen.
    \begin{requirements}
        \req[Kategorie]{F111} Eine Zeiterfassung ist mit einer Kategorie versehen.
        \req[Tätigkeit]{F112} Eine Zeiterfassung ist mit einer Tätigkeit verbunden.
        \req[Zeiterfassung ändern]{F113} Eine Zeiterfassung kann nachträglich geändert werden.
        \req[Alternative Zeiterfassung]{F114} Eine Zeiterfassung kann auch manuell ohne das Starten und Stoppen einer Zeit erfasst werden.
        \req[Zeiterfassung löschen]{F115} Eine erfasste Zeit kann vom Benutzer wieder gelöscht werden
    \end{requirements}

    \req[Gesetzliche Vorgaben]{F120}
    Die Zeiterfassung und der Stundenzettel folgen den gesetzlichen Bestimmungen.
    \begin{requirements}
        \req[Maximale Arbeitszeit]{F121} Die gesetzlich maximale Arbeitszeit kann bei einer Zeiterfassung nicht überschritten werden
        \req[Pausenzeiten]{F122} Eine Zeiterfassung kann nicht dem Stundenzettel hinzugefügt werden wenn für ihren Umfang gesetzliche Pausenzeiten nicht eingetragen wurden.
    \end{requirements}

    \req[Stundenzettel]{F130}
    Der Stundenzettel stellt eine Aufzeichnung der Arbeitstunden in einem vorgesetzten Zeitraum dar.
    \begin{requirements}
        \req[Sichtbarkeit]{F131} Der Betreuer und der Administrator können den Stundenzettel einsehen.
        \req[Stundenzettel abgeben]{F132} Ist der Benutzer mit seiner Zeiterfassung in einem Zeitraum fertig, so kann er den Stundenzettel abgeben.
        \req[abgegebene Stundenzettel]{F133} Der Betreuer und der Admin werden über abgegebene Stundenzettel Informiert
        \req[Stundenkonto]{F134} Im Stundenzettel ist die tatsächliche und die zu leistende Arbeitszeit sichtbar.
        \req[Betreuerkontrolle]{F135} Ein Stundenzettel kann vom Betreuer als Okay befunden werden.
    \end{requirements}

\end{requirements}

\subsection{Benutzer und Rechte}

\begin{requirements}
    \req[Benutzer]{F210}
    Ein Benutzer stellt den Standart User dar.
    \begin{requirements}
        \req[Stundenzettel]{F211} Ein Benutzer kann nur seinen eigenen Stundenzettel einsehen und verändern.
        \req[Warnungen]{F212} Ein Benutzer erhält nur Warnungen über seine eigenen Aktionen.
        \req[Erinnerungen]{F213} Ein Benutzer erhält nur Erinnerungen, die ihn selbst betreffen.
    \end{requirements}

    \req[Betreuer]{F220}
        Ein Betreuer ist für mehrere Benutzer zuständig.
        \begin{requirements}
            \req[Benutzer zuweisen]{F221} Einem Betreuer können Benutzer zugewiesen werden
            \req[Stundenzettel]{F222} Der Betreuer kann über seine übersicht alle ihm zugewiesenen Benutzer einsehen
            \req[Warnungen]{F223} Der Betreuer erhält Warnungen für alle Benutzer die ihn zugewiesen sind.
            \req[Erinnerungen]{F224} Der Betreuer erhält Erinnerungen für alle Benutzer die ihm zugewiesen sind.
            \req[Stundenzettel prüfen]{F225} Der Betreuer überprüft die agegebenen Stundenzettel der im zugewiesenen Benutzer.
        \end{requirements}

    \req[Admin]{F230}
        Ein Admin stellt das Verwaltungsorgan dar.
        \begin{requirements}
            \req[Benutzer erstellen]{F231} Der Admin kann neue Benutzer anlegen.
            \req[Benutzer editieren]{F232} Der Admin kann die Daten existierende Benutzer verändern.
            \req[Benutzer löschen]{F233} Der Admin kann einen existierenden Benutzer löschen.
            \req[Benutzer zuweisen]{F234} Der Admin kann Benutzer einem Betreuer zuweisen.
        \end{requirements}
\end{requirements}

