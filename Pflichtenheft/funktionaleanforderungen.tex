\section{Funktionale Anforderungen}

Die den Wunschkriterien zugeordneten Anforderungen sind mit einem "`+"' hinter der Indentifikationsnummer markiert.

\subsection{Stundenzettel und Zeiterfassung}

\begin{requirements}
    \req[Zeiterfassung]{F110}
    Ein User kann in seiner Hauptseite eine \em Zeiterfassung \em starten und stoppen.
    \begin{requirements}
        \req[Kategorie]{F111} Eine \em Zeiterfassung \em ist mit einer \em Kategorie \em versehen.
        \req[Tätigkeit]{F112} Eine \em Zeiterfassung \em ist mit einer \em Tätigkeit \em verbunden.
        \req[Zeiterfassung ändern]{F113} Eine \em Zeiterfassung \em kann nachträglich geändert werden.
        \req[Alternative Zeiterfassung]{F114} Eine \em Zeiterfassung \em kann auch manuell ohne das Starten und Stoppen einer Zeit erfasst werden.
        \req[Zeiterfassung löschen]{F115} Eine erfasste Zeit kann vom \em Benutzer \em wieder gelöscht werden.
    \end{requirements}

    \req[Gesetzliche Vorgaben]{F120}
    Die \em Zeiterfassung\em und der \em Stundenzettel \em folgen den gesetzlichen Bestimmungen.
    \begin{requirements}
        \req[Maximale Arbeitszeit]{F121} Die gesetzlich maximale Arbeitszeit kann bei einer \em Zeiterfassung\em nicht überschritten werden \em \(siehe §3 ArbZG\)\em.
        \req[Pausenzeiten]{F122} Eine \em Zeiterfassung\em kann nicht dem \em Stundenzettel\em hinzugefügt werden wenn für ihren Umfang gesetzliche Pausenzeiten nicht eingetragen wurden \em \(siehe §4 ArbZG\)\em.
        \req[Ruhezeiten]{F123} Die Einhaltung der Ruhezeiten wird erzwungen \em \(siehe §5 ArbZG\)\em.
        \req[Nachtarbeit]{F124} \em Administratoren\em können Nachtarbeit für einzelne Nutzer erlauben, allen anderen Nutzern ist Nachtarbeit nicht gestattet \em \(siehe §6 ArbZG\)\em.
        \req[Sonn- und Feiertage]{F125} Die Einhaltung der Sonn- und Feiertagsruhe wird erzwungen \em \(siehe §9 ArbZG\)\em.
		Dabei werden nur die Feiertage in Baden-Württemberg beachtet.
                Die Sonn- und Feiertagsruhe kann für einzelne \em Benutzer\em von \em Administratoren\em deaktiviert werden \em \(siehe §10 ArbZG\)\em, dabei wird dann §11 ArbZG beachtet.
        \req[Manipulation verhindern]{F126} Die gesetzlichen Vorgaben sind über eine Konfigurationsdatei anpassbar deren Prüfsumme im Code hinterlegt ist \em \(auch F410\)\em.
                Daher muss zum Ändern der gesetzlichen Vorgaben das Programm neu compiliert werden.
    \end{requirements}

    \req[Stundenzettel]{F130}
    Der \em Stundenzettel\em stellt eine Aufzeichnung der Arbeitstunden in einem vorgesetzten Zeitraum dar.
    \begin{requirements}
        \req[Sichtbarkeit]{F131} Der \em Betreuer\em und der \em Administrator\em können den Stundenzettel einsehen.
        \req[Stundenzettel abgeben]{F132} Ist der \em Benutzer\em mit seiner \em Zeiterfassung\em in einem Zeitraum fertig, so kann er den \em Stundenzettel abgeben\em.
        \req[abgegebener Stundenzettel]{F133} Ein \em abgegebener Stundenzettel\em kann vom \em Benutzer\em nichtmehr verändert werden.
        \req[abgegebene Stundenzettel]{F134} Der \em Betreuer\em und der \em Administrator\em werden über \em abgegebene Stundenzettel\em informiert
        \req[Stundenkonto]{F135} Im \em Stundenzettel\em ist die tatsächliche und die zu leistende Arbeitszeit sichtbar.
        \req[Betreuerkontrolle]{F136} Ein \em Stundenzettel\em kann vom \em Betreuer\em als \em geprüft\em eingestuft werden werden.
        \req[Betreuerkontrolle erfolgreich]{F137} Ein vom \em Betreuer\em als \em geprüft\em eingestufter \em Stundenzettel\em wird dem \em Administrator\em als \em abgegeben\em angezeigt.
        \req[Betreuerkontrolle fehlgeschlagen]{F138} Ein vom \em Betreuer\em abgelehnter \em Stundenzettel\em wird als \em nicht abgegeben\em markiert, vom \em Betreuer\em kommentiert, und der \em Benutzer\em darüber informiert.
        \req[Stundenzettel exportieren]{F139} Ein \em Administrator\em kann alle \em abgegebenen Stundenzettel\em eines Monats zum Drucken exportieren.
        \req[Drucken]{F13A}\em Benutzer\em, \em Betreuer \em und \em Administratoren \em können \em Stundenzettel \em drucken.
    \end{requirements}

\end{requirements}

\subsection{Benutzer und Rechte}

\begin{requirements}
    \req[Benutzer]{F210}
    Ein \em Benutzer \em stellt den Standart User dar.
    \begin{requirements}
        \req[Stundenzettel]{F211} Ein \em Benutzer \em kann nur seinen eigenen \em Stundenzettel \em einsehen und verändern.
        \req[Warnungen]{F212} Ein \em Benutzer \em erhält nur \em Warnungen \em über seine eigenen Aktionen.
        \req[Erinnerungen]{F213} Ein \em Benutzer \em erhält nur \em Erinnerungen \em, die ihn selbst betreffen.
    \end{requirements}

    \req[Betreuer]{F220}
        Ein \em Betreuer \em ist für mehrere \em Benutzer \em zuständig.
        \begin{requirements}
            \req[Benutzer zuweisen]{F221} Einem \em Betreuer \em können \em Benutzer \em zugewiesen werden
            \req[Stundenzettel]{F222} Der \em Betreuer \em kann über seine Übersicht alle ihm zugewiesenen \em Benutzer \em einsehen
            \req[Warnungen]{F223} Der \em Betreuer \em erhält \em Warnungen \em für alle \em Benutzer \em die ihn zugewiesen sind.
            \req[Erinnerungen]{F224} Der \em Betreuer \em erhält \em Erinnerungen \em für alle \em Benutzer \em die ihm zugewiesen sind.
            \req[Stundenzettel prüfen]{F225} Der \em Betreuer \em überprüft die \em abgegebenen Stundenzettel \em der im zugewiesenen \em Benutzer \em
        \end{requirements}

    \req[Administrator]{F230}
        Ein \em Admin \em stellt das Verwaltungsorgan dar.
        \begin{requirements}
            \req[Benutzer erstellen]{F231} Der \em Administrator \em kann neue \em Benutzer \em anlegen.
            \req[Benutzer editieren]{F232} Der \em Administrator \em kann die Daten existierender \em Benutzer \em verändern.
            \req[Benutzer löschen]{F233} Der \em Administrator \em kann einen existierenden \em Benutzer \em löschen.
            \req[Benutzer zuweisen]{F234} Der \em Administrator \em kann \em Benutzer \em einem \em Betreuer \em zuweisen.
            \req[Betreuer Ansicht]{F235} Der \em Administrator \em kann für jedes \em Team \em auch auf die \em Betreueransicht \em zugreifen.
        \end{requirements}
\end{requirements}

\subsection{Zeitüberwachung und Darstellung}
    \begin{requirements}
        \req[Graphische Darstellung Zeit]{F310+}
        \begin{requirements}
            \req[Übersicht Teams]{F311} Der \em Administrator \em soll die bisher benötigte Zeit aller \em Teams \em auf seiner \em Hauptseite \em einsehen können.
            \req[Übersicht Betreuer]{F312} Dem \em Betreuer \em soll die bisher benötigte Zeit aller im zugewiesenen \em Benutzer \em angezeigt werden.
            \req[Übersicht Benutzer]{F313} Der \em Benutzer \em soll seine bisher aufgewedete Zeit graphisch angezeigt werden.
            \req[Burn Rate Benutzer]{F314} Der \em Benutzer \em soll seine durchschnittliche wöchentliche Stundenrate \(Burn Rate\) angezeigt bekommen.
        \end{requirements}

        \req[Graphische Darstellung Stundenzettelabgabe]{F320}
        \begin{requirements}
            \req[Übersicht Teams]{F321} Der \em Administrator \em soll auf seiner \em Hauptseite \em die bisherigen \em Abgaben \em von \em Stundenzetteln \em angezeigt bekommen.
            \req[Übersicht Betreuer]{F322} Der \em Betreuer \em soll den bisherigen Abgabefortschritt seines \em Teams \em dargestellt bekommen.
        \end{requirements}

        \req[Darstellung von Daten]{F330}
        \begin{requirements}
            \req[Heatmap]{F331} \em Benutzer \em können einsehen an welchen Tagen die meiste Arbeitszeit geloggt wurde.
            \req[Punch Card]{F332} \em Benutzer \em können einsehen zu welchen Zeiten die meiste Arbeitszeit geloggt wurde.
        \end{requirements}

        \req[Tätigkeiten Darstellung]{F340}
        \begin{requirements}
            \req[Tätigkeiten Ranking]{F341} \em Benutzer \em können sich den Zeitverbrauch pro Tätigkeit, gesammelt über alle \em Benutzer \em, anzeigen lassen.
            \req[Tätigkeits Heatmap]{F342} \em Benutzer \em können sich eine Tätigkeits \em Heatmap \em, anzeigen lassen.
        \end{requirements}
        
        \req[Übersicht aller Daten]{F350}
        \begin{requirements}
             \req[Admin]{F351} \em Administratoren \em können alle geleisteten Stunden aller \em Benutzer \em graphisch einsehen.
             \req[Filter]{F352} Die graphische Anzeige erlaubt es nach \em Kategorien \em, \em Betreuern \em, \em Benutzer \em und Datum zu filtrieren.
        \end{requirements}

       \req[Framework zur Darstellung]{F360} Daten werden durch ein Statistik Framework visualisiert angezeigt.
    \end{requirements}

\subsection{Sonstige FA}
    \begin{requirements}
        \req[Sicherheit von Konfigurationen]{F410} Konfigurationsdateien sind durch einen Hash vor dem Editieren geschützt
        \req[Sicherheit der Datenbank]{F420} Die Datenbank wird durch einen Hash vor externem Editieren geschützt
        \begin{requirements}
            \req[Externe Veränderung der Datenbank]{F421} Sollte eine Veränderung stattgefunden haben, werden alle Nutzer über dieses externe Event informiert.
        \end{requirements}
    \end{requirements}

