\section{Funktionale Anforderungen}

Die den Wunschkriterien zugeordneten Anforderungen sind mit einem "`+"' hinter der Identifikationsnummer markiert.

\subsection{Stundenzettel und Zeiterfassung}

\begin{requirements}
    \req[Zeiterfassung]{F110}
    Ein \emph{Benutzer*} kann in seiner Hauptseite eine \emph{Zeiterfassung} starten und stoppen.
    \begin{requirements}
        \req[Kategorie]{F111} Eine \emph{Zeiterfassung} ist mit einer \emph{Kategorie} versehen.
        \req[Tätigkeit]{F112} Eine \emph{Zeiterfassung} ist mit einer \emph{Tätigkeit} verbunden.
        \req[Zeiterfassung ändern]{F113} Eine \emph{Zeiterfassung} kann nachträglich geändert werden.
        \req[Alternative Zeiterfassung]{F114} Eine \emph{Zeiterfassung} kann auch manuell ohne das Starten und Stoppen einer Zeit erfasst werden.
        \req[Zeiterfassung löschen]{F115} Eine erfasste Zeit kann vom \emph{Benutzer*} wieder gelöscht werden.
    \end{requirements}

    \req[Gesetzliche Vorgaben]{F120}
    Die \emph{Zeiterfassung} und der \emph{Stundenzettel} folgen den gesetzlichen Bestimmungen.
    \begin{requirements}
        \req[Maximale Arbeitszeit]{F121} Die gesetzlich maximale Arbeitszeit kann bei einer \emph{Zeiterfassung} nicht überschritten werden \emph{(siehe §3 ArbZG)}.
        \req[Pausenzeiten]{F122} Eine \emph{Zeiterfassung} kann nicht dem \emph{Stundenzettel} hinzugefügt werden wenn für ihren Umfang gesetzliche Pausenzeiten nicht eingetragen wurden \emph{(siehe §4 ArbZG)}.
        \req[Ruhezeiten]{F123} Die Einhaltung der Ruhezeiten wird erzwungen \emph{(siehe §5 ArbZG)}.
        \req[Nachtarbeit]{F124} \emph{Administratoren*} können Nachtarbeit für einzelne \emph{Benutzer*} erlauben, allen anderen \emph{Benutzern*} ist Nachtarbeit nicht gestattet \emph{(siehe §6 ArbZG)}.
        \req[Sonn- und Feiertage]{F125} Die Einhaltung der Sonn- und Feiertagsruhe wird erzwungen \emph{(siehe §9 ArbZG)}.
                Dabei werden nur die Feiertage in Baden-Württemberg beachtet.
                Die Sonn- und Feiertagsruhe kann für einzelne \emph{Benutzer*} von \emph{Administratoren*} deaktiviert werden \emph{(siehe §10 ArbZG)}, dabei wird dann \emph{§11 ArbZG} beachtet.
        \req[Manipulation verhindern]{F126} Die gesetzlichen Vorgaben sind über eine Konfigurationsdatei anpassbar deren Prüfsumme im Code hinterlegt ist \emph{(auch F410)}.
                Daher muss zum Ändern der gesetzlichen Vorgaben das Programm neu compiliert werden.
    \end{requirements}

\newpage
    \req[Stundenzettel]{F130}
    Der \emph{Stundenzettel} stellt eine Aufzeichnung der Arbeitsstunden in einem vorgesetzten Zeitraum dar.
    \begin{requirements}
        \req[Sichtbarkeit]{F131} Der \emph{Benutzer*}, der \emph{Betreuer*} und der \emph{Administrator*} können den Stundenzettel einsehen.
        \req[Stundenzettel abgeben]{F132} Ist der \emph{Benutzer*} mit seiner \emph{Zeiterfassung} in einem Zeitraum fertig, so kann er den \emph{Stundenzettel abgeben}.
        \req[abgegebener Stundenzettel]{F133} Ein \emph{abgegebener Stundenzettel} kann vom \emph{Benutzer*} nicht mehr verändert werden.
        \req[abgegebene Stundenzettel]{F134} Der \emph{Betreuer*} und der \emph{Administrator*} werden über \emph{abgegebene Stundenzettel} informiert.
        \req[Stundenkonto]{F135} Im \emph{Stundenzettel} ist die tatsächliche und die zu leistende Arbeitszeit sichtbar.
        \req[Betreuer*kontrolle]{F136} Ein \emph{Stundenzettel} kann vom \emph{Betreuer*} als \emph{geprüft} eingestuft werden.
        \req[Betreuer*kontrolle erfolgreich]{F137} Ein vom \emph{Betreuer*} als \emph{geprüft} eingestufter \emph{Stundenzettel} wird dem \emph{Administrator*} als \emph{abgegeben} angezeigt.
        \req[Betreuer*kontrolle fehlgeschlagen]{F138} Ein vom \emph{Betreuer*} abgelehnter \emph{Stundenzettel} wird als \emph{nicht abgegeben} markiert, vom \emph{Betreuer*} kommentiert, und der \emph{Benutzer*} darüber informiert.
        \req[Stundenzettel exportieren]{F139} Ein \emph{Administrator*} kann alle \emph{abgegebenen Stundenzettel} eines Monats zum Drucken exportieren.
        \req[Drucken]{F13A}\emph{Benutzer*}, \emph{Betreuer*} und \emph{Administratoren*} können \emph{Stundenzettel} drucken.
    \end{requirements}

\end{requirements}

\subsection{Benutzer* und Rechte}

\begin{requirements}
    \req[Benutzer*]{F210}
    Ein \emph{Benutzer*} stellt den Benutzer* ohne Sonderrechte dar.
    \begin{requirements}
        \req[Stundenzettel]{F211} Ein \emph{Benutzer*} kann nur seinen eigenen \emph{Stundenzettel} einsehen und verändern.
        \req[Warnungen]{F212} Ein \emph{Benutzer*} erhält nur \emph{Warnungen} über seine eigenen Aktionen.
        \req[Erinnerungen]{F213} Ein \emph{Benutzer*} erhält nur \emph{Erinnerungen}, die ihn selbst betreffen.
    \end{requirements}

    \req[Betreuer*]{F220}
        Ein \emph{Betreuer*} ist für mehrere \emph{Benutzer*} zuständig.
        \begin{requirements}
            \req[Benutzer* zuweisen]{F221} Einem \emph{Betreuer*} können \emph{Benutzer*} zugewiesen werden.
            \req[Stundenzettel]{F222} Der \emph{Betreuer*} kann über seine Übersicht alle ihm zugewiesenen \emph{Benutzer*} einsehen.
            \req[Warnungen]{F223} Der \emph{Betreuer*} erhält \emph{Warnungen} für alle \emph{Benutzer*} die ihn zugewiesen sind.
            \req[Erinnerungen]{F224} Der \emph{Betreuer*} erhält \emph{Erinnerungen} für alle \emph{Benutzer*} die ihm zugewiesen sind.
            \req[Stundenzettel prüfen]{F225} Der \emph{Betreuer*} überprüft die \emph{abgegebenen Stundenzettel} der ihm zugewiesenen \emph{Benutzer*}.
        \end{requirements}

    \req[Administrator*]{F230}
        Ein \emph{Administrator*} stellt das Verwaltungsorgan dar.
        \begin{requirements}
            \req[Benutzer* erstellen]{F231} Der \emph{Administrator*} kann neue \emph{Benutzer*} anlegen.
            \req[Benutzer* editieren]{F232} Der \emph{Administrator*} kann die Daten existierender \emph{Benutzer*} verändern.
            \req[Benutzer* löschen]{F233} Der \emph{Administrator*} kann einen existierenden \emph{Benutzer*} löschen.
            \req[Benutzer* zuweisen]{F234} Der \emph{Administrator*} kann \emph{Benutzer*} einem \emph{Betreuer*} zuweisen.
            \req[Betreuer*ansicht]{F235} Der \emph{Administrator*} kann für jedes \emph{Team} auch auf die \emph{Betreuer*ansicht} zugreifen.
        \end{requirements}
\end{requirements}

\subsection{Zeitüberwachung und Darstellung}
    \begin{requirements}
        \req[Graphische Darstellung Zeit]{F310+}
        \begin{requirements}
            \req[Übersicht Teams]{F311} Der \emph{Administrator*} soll die bisher benötigte Zeit aller \emph{Teams} auf seiner \emph{Hauptseite} einsehen können.
            \req[Übersicht Betreuer*]{F312} Dem \emph{Betreuer*} soll die bisher benötigte Zeit aller ihm zugewiesenen \emph{Benutzer*} angezeigt werden.
            \req[Übersicht Benutzer*]{F313} Der \emph{Benutzer*} soll seine bisher aufgewendete Zeit graphisch angezeigt werden.
            \req[Burn Rate Benutzer*]{F314} Der \emph{Benutzer*} soll seine durchschnittliche wöchentliche Stundenrate (Burn Rate) angezeigt bekommen.
        \end{requirements}

        \req[Graphische Darstellung Stundenzettelabgabe]{F320}
        \begin{requirements}
            \req[Übersicht Teams]{F321} Der \emph{Administrator*} soll auf seiner \emph{Hauptseite} die bisherigen \emph{Abgaben} von \emph{Stundenzetteln} angezeigt bekommen.
            \req[Übersicht Betreuer*]{F322} Der \emph{Betreuer*} soll den bisherigen Abgabefortschritt seines \emph{Teams} dargestellt bekommen.
        \end{requirements}

        \req[Darstellung von Daten]{F330}
        \begin{requirements}
            \req[Heatmap]{F331} \emph{Benutzer*} können einsehen an welchen Tagen die meiste Arbeitszeit geloggt wurde.
            \req[Punch Card]{F332} \emph{Benutzer*} können einsehen zu welchen Zeiten die meiste Arbeitszeit geloggt wurde.
        \end{requirements}

        \req[Tätigkeiten Darstellung]{F340}
        \begin{requirements}
            \req[Tätigkeiten Ranking]{F341} \emph{Benutzer*} können sich den Zeitverbrauch pro \emph{Tätigkeit}, gesammelt über alle \emph{Benutzer*}, anzeigen lassen.
            \req[Tätigkeits Heatmap]{F342} \emph{Benutzer*} können sich eine Tätigkeits-\emph{Heatmap}, anzeigen lassen.
        \end{requirements}
        
        \req[Übersicht aller Daten]{F350}
        \begin{requirements}
             \req[Administrator*]{F351} \emph{Administratoren*} können alle geleisteten Stunden aller \emph{Benutzer*} graphisch einsehen.
             \req[Filter]{F352} Die graphische Anzeige erlaubt es nach \emph{Kategorien}, \emph{Betreuer*}, \emph{Benutzer*} und Datum zu filtrieren.
        \end{requirements}

       \req[Framework zur Darstellung]{F360} Daten werden durch ein Statistik-Framework visualisiert angezeigt.
    \end{requirements}

\subsection{Sonstige FA}
    \begin{requirements}
        \req[Sicherheit von Konfigurationen]{F410} Bestimmte Konfigurationsdateien sind durch eine im Quellcode hinterlegte Prüfsumme vor dem Editieren geschützt.
        \req[Sicherheit der Datenbank]{F420+} Die Datenbank wird durch eine Prüfsumme vor externem Editieren geschützt.
        \begin{requirements}
            \req[Externe Veränderung der Datenbank]{F421} Sollte eine Veränderung stattgefunden haben, werden alle Nutzer über dieses externe Event informiert.
        \end{requirements}
    \end{requirements}

