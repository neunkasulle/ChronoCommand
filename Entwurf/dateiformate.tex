\section{Dateiformate}
\subsection{Dateiformat zur Definition von gesetzlichen Feiertagen}
In einer Datei "`Feiertage.dat"' können die gesetzlichen Feiertage des entsprechenden Bundeslandes konfiguriert werden.
Ausgeliefert wird das Programm mit den gesetzlichen Feiertagen für Baden-Württemberg.
Wenn die Datei geändert wurde, muss das Programm danach neu kompiliert werden, da im Quellcode eine Prüfsumme der Datei hinterlegt ist um Manipulation zu erschweren.
Die Datei ist ASCII-kodiert.

In jeder Zeile steht der Name eines Feiertags gefolgt von einem Doppelpunkt und dem Datum.
Das Datum kann auf zwei Arten hinterlegt werden:
\begin{itemize}
    \item Das Datum im Format TT.MM., also zum Beispiel 05.01.
    \item Als Anzahl der Tage vor oder nach dem Osterdatum, im Format +N bzw -N, also zum Beispiel +60
\end{itemize}

\subsection{Datenbankanbindung}
In der Datei "`hibernate.cfg.xml"' wird die Datenbankanbindung mittels der Variablen connection.driver\_class, connection.url, connection.username und connection.password definiert.
Diese Datei muss
Die Syntax der Datei entspricht den Vorgaben von Hibernate ORM 5.0.

\subsection{Authentifizierung via LDAP}
Falls die Authentifizierung via LDAP erwünscht ist, muss die Anbindung der "`Shiro.ini"' konfiguriert werden.
Die Syntax der Datei entspricht den Vorgaben von Shiro 1.2.4.
