\section{Fehlerbeschreibung}

\subsection{Behobene Fehler}

\subsubsection{Fehler 1:} %make icons white
\textbf{Fehlersymptom:}
	\begin{itemize}
		\item Die Knöpfe für "`Hauptseite"' und "`Logout"' waren schlecht zu erkennen.
	\end{itemize}
\textbf{Fehlergrund:}
	\begin{itemize}
		\item Die Knöpfe waren schwarz, so wie die Hintergrundfarbe der Webseite.
	\end{itemize}
\textbf{Fehlerbehebung:}
	\begin{itemize}
		\item Die Farbe der Knöpfe wurde zu weiß geändert.
	\end{itemize}

\subsubsection{Fehler 2:} %Receive no weekly reminder
\textbf{Fehlersymptom:}
	\begin{itemize}
		\item In manchen Wochen wurde keine Erinnerung zum letzten Arbeitszeiteintrag verschickt.
	\end{itemize}
\textbf{Fehlergrund:}
	\begin{itemize}
		\item Die Abfrage nach dem letzten Arbeitszeiteintrag hat nach Tagen, anstatt nach Daten, verglichen.
	\end{itemize}
\textbf{Fehlerbehebung:}
	\begin{itemize}
		\item Es werden nun ganze Daten verglichen und nicht mehr nur Tage.
	\end{itemize}

\subsubsection{Fehler 3:} %Send monthly reminder to users without timesheet
\textbf{Fehlersymptom:}
	\begin{itemize}
		\item Benutzer* ohne Stundenzettel, welche aber arbeiten müssen, erhielten keine monatiche Erinnerung.
	\end{itemize}
\textbf{Fehlergrund:}
	\begin{itemize}
		\item Es wurde nur eine Erinnerung an Leute mit Stundenzettel verschickt.
	\end{itemize}
\textbf{Fehlerbehebung:}
	\begin{itemize}
		\item Alle Benutzer* mit Arbeitszeit erhalten eine Erinnerung.
	\end{itemize}

\subsubsection{Fehler 4:} %UI bug in admin or supervisor view
\textbf{Fehlersymptom:}
	\begin{itemize}
		\item Beim Klick auf "`Show all Users"' wurde die Anzeige mehrfach nebeneinander angezeigt.
	\end{itemize}
\textbf{Fehlergrund:}
	\begin{itemize}
		\item Das Template zur Anzeige wurde nicht verworfen und ist so zusammen mit der neuen Anzeige erschienen.%TODO stimmt das so?
	\end{itemize}
\textbf{Fehlerbehebung:}
	\begin{itemize}
		\item Das Template wird nun korrekt verwendet.
	\end{itemize}
	
\subsubsection{Fehler 5:} %limit possible characters for usernames
\textbf{Fehlersymptom:}
	\begin{itemize}
		\item Der Pool der möglichen Zeichen eines Benutzer*namens war nicht beschränkt.
	\end{itemize}
\textbf{Fehlergrund:}
	\begin{itemize}
		\item Es wurde keine Abfrage auf Sonderzeichen im Namen gemacht.
	\end{itemize}
\textbf{Fehlerbehebung:}
	\begin{itemize}
		\item Es wurde eine Abfrage nach Sonderzeichen hinzugefügt.
	\end{itemize}

\subsubsection{Fehler 6:} %refactor whole time sheet new and close
\textbf{Fehlersymptom:}
	\begin{itemize}
		\item Stundenzettel enthielten fehlerhafte Daten.
	\end{itemize}
\textbf{Fehlergrund:}
	\begin{itemize}
		\item Es gabe mehrere Methoden um einen Stundenzettel zu manipulieren die verschiedene Signaturen aufwiesen.
	\end{itemize}
\textbf{Fehlerbehebung:}
	\begin{itemize}
		\item Die Schnittstellen zum Manipulieren der Stundenzettel wurden umgeschrieben.
	\end{itemize}

\subsubsection{Fehler 7:} %Supervisors and Administrators can look at other timesheets
\textbf{Fehlersymptom:}
	\begin{itemize}
		\item Ein Supervisor* konnte auch andere Stundenzettel ansehen und nicht nur die, der ihm zugewiesenen Benutzer*.
	\end{itemize}
\textbf{Fehlergrund:}
	\begin{itemize}
		\item Dieser Fall wurde bei der Rechteverteilung nicht berücksichtigt.
	\end{itemize}
\textbf{Fehlerbehebung:}
	\begin{itemize}
		\item Die Abfrage der Rechte zum Einsehen des Stundenzettels wurde geändet.
	\end{itemize}

\subsubsection{Fehler 8:} %get new session when logging out
\textbf{Fehlersymptom:}
	\begin{itemize}
		\item Wenn sich ein Benutzer* ausloggte, konnte er sich nicht sofort wieder anmelden.
	\end{itemize}
\textbf{Fehlergrund:}
	\begin{itemize}
		\item 
	\end{itemize}
\textbf{Fehlerbehebung:}
	\begin{itemize}
		\item 
	\end{itemize}
	
\subsubsection{Fehler 9:}%show success notification when user edits themselves
\textbf{Fehlersymptom:}
	\begin{itemize}
		\item Der Benutzer* wird nicht benachrichtigt wenn er seine persönlichen Daten erfolgreich ändert.
	\end{itemize}
\textbf{Fehlergrund:}
	\begin{itemize}
		\item Es existiert keine Benachrichtigung, welche zu diesem Fall passt.
	\end{itemize}
\textbf{Fehlerbehebung:}
	\begin{itemize}
		\item Es wurde eine entsprechende Benachrichtigung hinzugefügt.
	\end{itemize}

\subsubsection{Fehler 10:}%Can't edit start/end times in existing time record
\textbf{Fehlersymptom:}
	\begin{itemize}
		\item Beim Editieren von Zeiteinträgen können die Start- und Endzeit nicht verändert werden.
	\end{itemize}
\textbf{Fehlergrund:}
	\begin{itemize}
		\item Die Werte, welche geändert werden sollten, konnten nicht gespeichert werden.
	\end{itemize}
\textbf{Fehlerbehebung:}
	\begin{itemize}
		\item Das Backend wurde angepasst um Zeiteinträge richtig zu ändern.
	\end{itemize}

\subsubsection{Fehler 11:}%Support deleting timerecords
\textbf{Fehlersymptom:}
	\begin{itemize}
		\item Arbeitszeiteinträge konnten nicht gelöscht werden.
	\end{itemize}
\textbf{Fehlergrund:}
	\begin{itemize}
		\item Das Backend hat das Löschen nicht unterstützt.
	\end{itemize}
\textbf{Fehlerbehebung:}
	\begin{itemize}
		\item Es wurden entsprechende Methoden hinzugefügt, die Arbeitszeiteinträge löschen.
	\end{itemize}

\subsubsection{Fehler 12:}%Admin edit user: disabled selecting an other user while editing
\textbf{Fehlersymptom:}
	\begin{itemize}
		\item Wenn ein Administrator einen User bearbeitet, konnte er währenddessen einen anderen User auswählen und dessen Daten wurden dann stattdessen überschrieben.
	\end{itemize}
\textbf{Fehlergrund:}
	\begin{itemize}
		\item Tabelle wurde beim User editieren nicht deaktiviert.
	\end{itemize}
\textbf{Fehlerbehebung}
	\begin{itemize}
		\item Tabelle wird nun vor dem editieren deaktiviert und danach wieder aktiviert.
	\end{itemize}
	
\subsubsection{Fehler 13:}%Adminview: update table when saving a user
\textbf{Fehlersymptom:}
	\begin{itemize}
		\item Wenn ein Administrator einen User bearbeitet und speichert, erscheinen die neuen Daten nicht direkt in der Tabelle.
	\end{itemize}
\textbf{Fehlergrund:}
	\begin{itemize}
		\item Die entsprechende Methode wurde nicht aufgerufen.
	\end{itemize}
\textbf{Fehlerbehebung:}
	\begin{itemize}
		\item Die Methode wird nun aufgerufen.
	\end{itemize}
	
\newpage

\subsection{Nichtbehobene Fehler}

\subsubsection{Fehler 14:}%Bei timesheet lock überprüfen ob nicht zu wenig gearbeitet wurde
\textbf{Fehlersymptom:}
	\begin{itemize}
		\item Ein Stundenzettel kann beim Einreichen weniger Arbeitszeit haben als notwendig.
	\end{itemize}
\textbf{Fehlergrund:}
	\begin{itemize}
		\item Es wird keine Abfrage dafür durchgeführt.
	\end{itemize}
	
\subsubsection{Fehler 15:}%Set message for timesheet when rejecting
\textbf{Fehlersymptom:}
	\begin{itemize}
		\item Beim Prüfen eines Stundenzettels vom Betreuer*, kann kein Grund für die Ablehnung angegeben werden.
	\end{itemize}
\textbf{Fehlergrund:}
	\begin{itemize}
		\item Wurde aus Zeitgründen nicht implementiert.
	\end{itemize}

\subsubsection{Fehler 16:}%Make table full width of page
\textbf{Fehlersymptom:}
	\begin{itemize}
		\item Die Tabelle mit den Arbeitszeiteinträgen der Stundenzettel wird nurin einer bestimmten Größe dargestellt. Bei größeren Bildschirmen wird sie kleiner als nötig dargestellt.
	\end{itemize}
\textbf{Fehlergrund:}
	\begin{itemize}
		\item Die Größe der Tabelle ist statisch.
	\end{itemize}
