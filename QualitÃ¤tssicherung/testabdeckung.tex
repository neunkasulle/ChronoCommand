\section{Testabdeckung}
Es wurde eine Testabdeckung von 100\% der Klasse, 96,2\% der Methoden, 84,7\% der Zeilen erreicht. Allerdings wurden nur die Packete \emph{Control}, \emph{Model} und \emph{Security} getestet.\\\\
Zum Vergleich wurde in den Test der \emph{Implementierungsphase} eine Testabdeckung der Klassen von 36,7\%, der Methoden von 35,3\% und eine Zeilenabdeckung von 27,5\% erreicht.
\\\\
Das Packet \emph{View} wurde nicht getestet, da es nur die Elemente des graphischen Interfaces enthält und somit nicht getestet werden kann ob z.B. die richtige Reaktion erfolge, wenn ein bestimmte Knopf gedrückt wurde.\\

Zum Testen wurde JUnit verwendet.

\subsection{Control}
Im Packet \emph{Control} wurde eine Klassenabdeckung von 100\%, mit einer Methodenabdeckung von 94,9\% und einer Zeilenabdeckung von 80,1\%, erreicht.\\\\
Nach der \emph{Implementierungsphase} bestand die Abdeckung aus 45,5\% Klassenabdeckung, mit 42,7\% Methoden- und 45,4\% Zeilenabdeckung.

\subsection{Model}
Es wurde eine Klassenabdeckung von 100\%, eine Methodenabdeckung von 96,8\% und eine Zeilenabdeckung von 87,9\% im Packet \emph{Model} erreicht.\\\\
Die Testabdeckung nach der \emph{Implementierungsphase} unterscheidet sich nur in der Methoden- (65\%) und Zeilenabdeckung(58,6\%).

\subsection{Security}
Da es im Packet \emph{Security} nur eine Klasse gibt, wurde eine Abdeckung von 100\% der Klassen und Methoden erreicht. Die Zeilenabdeckung ist 90,9\%.\\\\
Die Zeilenabdeckung nach der \emph{Implementierungsphase} betrug 81,8\%, die Klassen- und Methodenabdeckung war gleich.
