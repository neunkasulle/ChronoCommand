\section{Testszenarien}


\textbf{T100: Ein Benutzer* meldet sich an.} 

$\XBox$ T111: Aufrufen der Internetseite. \\
$\XBox$ T112: Logindaten eingeben und Anmeldeknopf drücken. \\

\textbf{T120: Eine neue Zeiterfassung starten/stoppen.}

T100: Ein Benutzer* meldet sich an. \\
T121: Starten einer neuen Zeiterfassung. \\
T122: Stoppen und speichern einer Zeiterfassung. \\

\textbf{T130: Ein Benutzer* erhält eine Warnung wegen einem nicht abgegebenen Stundenzettel.}

T100: Ein Benutzer* meldet sich an. \\
T130: Der Benutzer* erhält nach der Anmeldung eine Warnung für einen nicht abgegebenen Stundenzettel. \\

\textbf{T140: Ein Betreuer* prüft einen neu abgegebenen Stundenzettel.}

T100: Ein Betreuer* meldet sich an. \\
T141: Der Betreuer* bekommt die Meldung für einen abgegebenen Stundenzettel. \\
T142: Der Betreuer* prüft den Stundenzettel und klickt auf "Korrekt" oder "Nicht Korrekt". \\

\textbf{T150: Editieren der Tätigkeit in einem nicht abgegebenem Stundenzettel.}

T100: Ein Benutzer* meldet sich an. \\
T151: Der Benutzer* klickt auf "Editieren" bei einer eingetragenen Zeit und kann den Eintrag bearbeiten. \\
T152: Der Benutzer* bearbeitet die Tätigkeit und drückt auf "Speichern". \\

\textbf{T160: Anmeldung mit einem gültigem Session Cookie.}

T111: Aufrufen der Internetseite. \\
T160: Der Benutzer* hat ein Session Cookie und wird angemeldet. \\

\textbf{T170: Ein Benutzer* verletzt gesetzliche Vorgaben.}

T100: Ein Benutzer* meldet sich an. \\
T121: Starten einer neuen Zeiterfassung. \\
T170: Der Benutzer* erhält, während einer Zeiterfassung, eine Warnung, weil er die gesetzlichen Vorgaben verletzt. \\

\newpage

\textbf{T180: Ein Betreuer* sieht Stundendaten ein.}

T100: Ein Benutzer* meldet sich an. \\
T180: Der Betreuer* klickt auf die Schaltfläche "Stundenzettel einsehen" und erhält die Stundendaten der ihm zugewiesenen Benutzer*. \\
        
\textbf{T190: Manuelles Eintragen von Arbeitszeit.}

T110: Ein Benutzer* meldet sich an. \\
T191: Der Benutzer* klickt auf die Schaltfläche "Zeit manuell eintragen". \\
T192: Der Benutzer* trägt die Daten der Zeit ein und speichert. Die erfasste Zeit wird gespeichert. \\
T193: Der Benutzer* erhält eine Warnung, da die Angaben nicht vollständig sind. \\

\textbf{T200: Löschen einer erfassten Zeit.}

T100: Ein Benutzer* meldet sich an. \\
T201: Durch Klicken von "Löschen" bei einer eingetragenen Zeit wird diese gelöscht. \\
    
\textbf{T210: Stoppen einer unvollständigen Zeiterfassung.}

T100: Ein Benutzer* meldet sich an. \\
T211: Beim Stoppen einer Zeiterfassung wird eine Warnung ausgegeben, da keine Tätigkeit eingetragen ist. \\
T212: Beim Stoppen einer Zeiterfassung wird eine Warnung ausgegeben, da keine Kategorie eingetragen ist. \\
            
\textbf{T220:  Ein Administrator* schaut sich alle Daten an.}

T100: Ein Administrator* meldet sich an. \\
T221: Der Administrator* navigiert zur Übersicht aller Daten und erhält eine graphische Übersicht. \\
T222: Der Administrator* wählt eine Kategorie aus und erhält alle geleisteten Stunden in dieser Kategorie.
