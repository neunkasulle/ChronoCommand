\section{Testszenarien}
Die nachfolgenden Testszenarien bestehen aus Testfällen, welche im Pflichtenheft \linebreak definiert wurden. \\
\colorbox{green}{OK} steht für einen erfolgreichen Test und \colorbox{red}{FL} für einen fehlgeschlagegenen Test.


\subsection{Testszenario 1 - Anmeldung und neue Zeiterfassung}
\emph{T100} Ein Benutzer* meldet sich an und \emph{T120} startet eine neue Zeiterfassung. Anschließend stoppt er die Zeiterfassung.
\\\\
\colorbox{green}{OK} \textbf{T111} Aufrufen der Internetseite. \\
\colorbox{green}{OK} \textbf{T112} Logindaten eingeben und Anmeldeknopf drücken. \\
\colorbox{green}{OK} \textbf{T121} Starten einer neuen Zeiterfassung. \\
\colorbox{green}{OK} \textbf{T122} Stoppen und speichern einer Zeiterfassung.

\subsection{Testszenario 2 - Nicht abgegebener Stundenzettel}
\emph{T130} Es wird eine Warnung ausgegeben, weil ein Benutzer* seinen Stundenzettel nicht abgegeben hat.
\\\\
\colorbox{green}{OK} \textbf{T100} Anmeldung. \\
\colorbox{green}{OK} \textbf{T130} Es wird eine Warnung ausgegeben.

\subsection{Testszenario 3 - Stundenzettelprüfung}
\emph{T140} Ein Betreuer* bekommt eine Meldung für einen abgegebenen Stundenzettel. Er prüft ihn und entscheidet ob er korrekt oder nicht korrekt ist.
\\\\
\colorbox{green}{OK} \textbf{T100} Anmeldung. \\
\colorbox{orange}{  \ ? \  } \textbf{T141} Meldung für einen abgegebenen Stundenzettel. \\
\colorbox{orange}{  \ ? \  } \textbf{T142} Der Betreuer* klickt auf "`Korrekt"'. \\
\colorbox{orange}{  \ ? \  } \textbf{T142} Der Betreuer* klickt auf "`Nicht Korrekt"'.

\subsection{Testszenario 4 - Editieren einer erfassten Zeit}
\emph{T150} Ein Benutzer* editiert, bei einem nicht abgegebenen Stundenzettel, eine eingetragene Zeit. Danach speichert er die editierte Zeit.
\\\\
\colorbox{green}{OK} \textbf{T100} Anmeldung. \\
\colorbox{green}{OK} \textbf{T151} Der Benutzer* klickt auf "`Editieren". \\
\colorbox{green}{OK} \textbf{T152} Editieren und "`Speichern"'.

\subsection{Testszenario 5 - Session Cookie}
\emph{T160} Ein Benutzer* hat durch eine frühere Anmeldung ein gültiges Session Cookie erhalten. Er ruft die Internetseite erneut auf und muss seine Anmeldedaten nicht erneut eingeben.
\\\\
\colorbox{green}{OK} \textbf{T111} Aufrufen der Internetseite. \\
\colorbox{green}{OK} \textbf{T160} Der Benutzer* hat ein Session Cookie und wird angemeldet.

\subsection{Testszenario 6 - Betreuer* sieht Stundendaten ein}
\emph{T180} Ein Betreuer* klickt auf die Schaltfläche "`Stundenzettel einsehen"'. Er erhält die Übersicht über die Stundendaten des Stundenzettels, vom jeweiligen Benutzer*.
\\\\
\colorbox{green}{OK} \textbf{T100} Anmeldung. \\
\colorbox{green}{OK} \textbf{T180} Klick auf "`Stundenzettel einsehen"', die Übersicht erscheint.

\subsection{Testszenario 7 - Hinzufügen und löschen von Arbeitszeit}
\emph{T190} Ein Arbbeitszeit wird manuell hinzugefügt. \emph{T200} Eine eingeragene Zeit wird, durch klicken des "`Löschen"-Knopfs,' gelöscht.
\\\\
\colorbox{green}{OK} \textbf{T100} Anmeldung. \\
\colorbox{green}{OK} \textbf{T191} Der Benutzer* klickt auf die Schaltfläche "`Zeit manuell eintragen"'. \\
\colorbox{green}{OK} \textbf{T192} Eintragen der entsprechenden Daten. \\
\colorbox{green}{OK} \textbf{T193} Der Benutzer* erhält eine Warnung, da die Angaben nicht vollständig sind. \\
\colorbox{green}{OK} \textbf{T201} Löschen einer erfassten Zeit.

\subsection{Testszenario 8 - Unvollständige Zeiterfassung stoppen}
\emph{T210} Es läuft eine Zeiterfassung. Der Benutzer* möchte die Zeiterfassung stoppen, erhält allerdings eine Warnung, da die Zeiterfassung unvollständig ist.
\\\\
\colorbox{green}{OK} \textbf{T100} Anmeldung. \\
\colorbox{green}{OK} \textbf{T121} Starten einer neuen Zeiterfassung. \\
\colorbox{green}{OK} \textbf{T211} Warnung für fehlende Tätigkeit. \\
\colorbox{green}{OK} \textbf{T212} Warnung für fehlende Kategorie.
