\section{Einleitung}

Dieses Dokument beschreibt die Qualitätssicherungsphase des PSE-Projektes \linebreak \emph{ChronoCommand}. Die Qualitätssicherung soll die Qualität des Produktes feststellen, ob der Code korrekt und stabil ist, Fehler finden und diese anschließend beheben.
\\\\
Die Qualitätssicherung umfasst folgende Punkte:
\begin{itemize}
	\item \textbf{Testszenarien}
\end{itemize}
Bei den Testszenarien werden die Testfälle aus dem Pflichtenheft am laufenden Programm getestet. Dabei soll eine grobe Übersicht über die Funktionalität der Software entstehen.

\begin{itemize}
	\item \textbf{Testabdeckung}
\end{itemize}
Für die Qualitätssicherungsphase üblichen Tests, gibt die Testabdeckung einen Überblick. Es wurde versucht eine hohe Abdeckung zu erreichen, um möglichst viele Fehler zu finden.

\begin{itemize}
	\item \textbf{Fehlerbeschreibung}
\end{itemize}
Alle, in den vorherigen Punkten, gefundenen Fehler und deren Behebung sind in der Fehlerbeschreibung beschrieben. 
