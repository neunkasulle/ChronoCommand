\section{Testszenarien}


\subsection{T100: Ein Benutzer* meldet sich an.}
T111: Aufrufen der Internetseite.
T112: Logindaten eingeben und Anmeldeknopf drücken.

\subsection{T120: Ein angemeldeter Benutzer* startet/stoppt eine Zeiterfassung.}
T121: Starten einer neuen Zeiterfassung.
T122: Stoppen und speichern einer Zeiterfassung

\subsection{T130: Ein Benutzer* erhält eine Warnung wegen einem nicht abgegebenen Stundenzettel.}
T110: Anmelden.
T130: Der Benutzer* erhält nach der Anmeldung eine Warunung für einen nicht abgegebenen Stundenzettel.

\subsection{T140: Ein Betreuer* prüft einen neu abgegebenen Stundenzettel.}
T141: Ein Betreuer* bekommt die Meldung für einen abgegebenen Stundenzettel.
T142: Der Betreuer* prüft den Stundenzettel und klickt auf "Korrekt" oder "Nicht Korrekt".

\subsection{T150: Ein Benutzer* editiert die Tätigkeit in einem nicht abgegebenem Stundenzettel.}
T151: Der Benutzer* klickt auf "Editieren" bei einer eingetragenen Zeit und kann den Eintrag bearbeiten.
T152: Der Benutzer* bearbeitet die Tätigkeit und drückt auf "Speichern".

\subsection{T160: Der Benutzer* wählt die Seite an und besitzt ein gültiges Session Cookie.}
T111: Aufrufen der Internetseite.
T160: Der Benutzer* hat ein Session Cookie und wird angemeldet.

\subsection{T170: Ein Benutzer* verletzt gesetzliche Vorgaben.}
T170: Ein Benutzer* erhält, während einer Zeiterfassung, eine Warnung, weil er die gesetzlichen Vorgaben verletzt.
            
\subsection{T180: Ein Betreuer* sieht Stundendaten ein.}
T180: Ein Betreuer* klickt auf die Schaltfläche "Stundenzettel einsehen" und erhält die Stundendaten der ihm zugewiesenen Benutzer*.
        
\subsection{T190: Ein Benutzer* trägt manuell eine Zeit ein.}
T110: Ein Benutzer* meldet sich an.
T191: Der Benutzer* klickt auf die Schaltfläche "Zeit manuell eintragen".
T192: Der Benutzer* trägt die Daten der Zeit ein und speichert. Die erfasste Zeit wird gespeichert.
193: Der Benutzer* erhält eine Warnung, da die Angaben nicht vollständig sind.

\subsection{T200: Ein Benutzer* löscht eine erfasste Zeit.}
T201: Durch Klicken von "Löschen" bei einer eingetragenen Zeit wird diese gelöscht.
    
\subsection{T210: Ein Benutzer* stoppt eine unvollständige Zeiterfassung.}
211: Beim Stoppen einer Zeiterfassung wird eine Warnung ausgegeben, da keine Tätigkeit eingetragen ist.
212: Beim Stoppen einer Zeiterfassung wird eine Warnung ausgegeben, da keine Kategorie eingetragen ist.
            
\subsection{T220: Ein Administrator* schaut sich alle Daten an.}
221: Ein Administrator* navigiert zur Übersicht aller Daten und erhält eine graphische Übersicht.
222: Der Administrator* wählt eine Kategorie aus und erhält alle geleisteten Stunden in dieser Kategorie.
