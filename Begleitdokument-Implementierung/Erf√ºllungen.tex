\section{Kriterien}
Im Folgenden wird aufgeführt, welche Muss- und Wunschkriterien erfüllt wurden und welche nicht. Desweiteren wird bei nicht erfüllten Kriterien beschreiben, was die Gründe dafür sind.

    \begin{itemize}
    \item \subsection{Musskriterien}
    \begin{itemize}
      \item Starten und Stoppen von einer zeiterfassung: \emph{erfüllt}
      \item Zeiten nachträglich erfassen: \emph{erfüllt} 
      \item Zeiten nachträglich ändern: \emph{erfüllt}
      \item Erfasste Zeiten können Tätigkeiten zugeordnet werden: \emph{erfüllt}
      \item Erfasste Zeiten können Kategorien zugeordnet werden: \emph{erfüllt}
      \item Warnungen wenn Zeiten nicht eingetragen sind: \emph{erfüllt}
      \item Möglichkeit die erfassten Zeiten an den Administrator* und Betreuer* zu übersenden: \emph{erfüllt}
	    \item Möglichkeit den Stundenzettel an den Administrator* und Betreuer* zu übersenden: \emph{erfüllt}
	    \item Warnungen wenn gesetzlich Pausen genommen werden müssen: \emph{nicht erfüllt, da die Datenerhebung für Pause signifikant komplizierter ist und uns dafür die Zeit nicht mehr gereicht hat}
	    \item Warnungen wenn zu erfassten Zeiten keine Tätigkeit zugeordnet ist: \emph{erfüllt}
	    \item Warnungen wenn zu erfassten Zeiten keine Kategorie zugeordnet ist: \emph{erfüllt}
	    \item Vergangene Zeiten sind einsehbar: \emph{erfüllt}
    	\item Hinzufügen von Accounts: \emph{erfüllt}
    	\item Löschen von Accounts: \emph{nicht erfüllt, da keine Accounts gelöscht werden. Nichtbenutze Accounts werden deaktiviert.}
    	\item Ändern von Accounts: \emph{erfüllt}
    	\item Bestimmte Benutzer* sind Administratoren*: \emph{erfüllt}
    	\item Bestimmte Benutzer* sind Betreuer*: \emph{erfüllt}
    	\item Administratoren* legen fest, welche Benutzer* von welchem Betreuer* betreut werden: \emph{erfüllt}
    	\item Backups werden regelmäßig angefertigt: \emph{(nicht) erfüllt, da dies die Aufgabe der Administratoren ist}
    	\item Es kann zwischen LDAP und lokalen Accounts zur Benutzer*verwaltung gewählt werden: \emph{nicht erfüllt, aus Zeitgründen}
    	\item Zeiten sollen durch eine graphische Übersicht visualisiert werden (Heatmap, Punch Card): \emph{nicht erfüllt}
  \end{itemize}
    
    
  \newpage
    \item \subsection{Wunschkriterien}
    \begin{itemize}
	    \item Mehrere Frontendimplementierungen sind möglich: \emph{erfüllt}
    	\item Zeitvorhersagen anhand vergangener Arbeitszeit: \emph{nicht erfüllt, wegen Zeitmangen und zu wenog Erfahrung mit dem Thema}
    	\item Überwachung der Arbeitszeit auch ohne abgegebene Zeiten möglich: \emph{nicht erfüllt, da die Umsetzung zu komplex ist} 
    	\item Erfassung der Tätigkeiten während der Arbeitszeit: \emph{nicht erfüllt, da die Umsetzung zu komplex ist}
    	\item Möglichkeit, die IP-Range, von der aus sich Administratoren* einloggen können, zu beschränken: \emph{nicht erfüllt, aus Zeitgründen}
    	\item Kalendarische Übersicht erfasster Zeiten: \emph{nicht erfüllt, aus Zeitgründen}
    	\item Über eine Toolbar Zeiten erfassen und eine Tätigkeit zuordnen: \emph{erfüllt}
  \end{itemize}
  \end{itemize}



      
