\section{Einleitung}

Dies ist das Begleitdokument zur Implementierungsphase von \emph{ChronoCommand} der Gruppe 1. Bei \emph{ChronoCommand} handelt
es sich um ein Programm zur digitalen Zeiterfassung welches durch Hilfswissenschaftler (im Folgenden HiWi genannt) oder andere
geringfügig beschäftigter Mitarbeiter verwendet werden kann. Durch \emph{ChronoCommand} fällt der zusätzliche Schreibaufwand der HiWis weg, welche ohne
\emph{ChronoCommand} von Hand aufschreiben müssen, wann sie woran wielange gearbeitet haben.Im nachfolgenden Text wird 
beschrieben, inwieweit wir von dem Entwurf abgewichen sind und beinhaltet eine allgemeine Beschreibungs unseres Vorgehens.
Als Erstes wird aufgeführt, welche Veränderungen wir am Klassendiagramm der Entwurfsphase wir durchgeführt haben. 
Dies wird anhand eines aktualisierten Klassendiagramms und anhand der im Entwurfsdokument beschreibenen Sequenzdiagramme 
beschreiben. Es wird eine Übersicht darüber geben, welche Klassen hinzugefügt wurden und welche Klassen hinfällig wurden. Des 
Weiteren wird aufgezeigt, welche Methoden wir nicht vorhergesehen haben, jedoch trotzdem benötigt wurden und welche Methode 
entweder durch die Umstellung bzw. Änderung der Klassen weggefallen sind oder uns von externen Frameworks abgenommen wurden.
Als Nächstes wird eine Übersicht gegeben, welche Musskriterien aus dem Pflichtenheft wird implementiert haben und welche nicht
machbar waren. Außerdem wird aufgeführt, welche Wunschkriterien wir einbauen konnten und welche es leider nicht in die 
Implementierung geschafft haben. Darauf folgend wird beschrieben, welche Abweichungen vom ursprünglichen Zeitplan es gab 
zusammen mit einer Beschreibung warum sich diese Veränderungen ergeben haben. Daran anschließend wird es eine Aufzählung und Erklärung zu allen vorhandenen Unit-Tests geben und warum diese notwendig sind.
