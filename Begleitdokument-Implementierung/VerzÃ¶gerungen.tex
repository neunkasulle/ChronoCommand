\subsection{Allgemeines}
Einer der größten Faktoren durch den sich die Implementierung verzögert hat, ist die Unerfahrenheit des Teams. 
Hier gibt es mehrere Stellen, an denen sich dies gezeigt hat. Allgemein hat niemand aus unserem Team Erfahrung 
mit Web-Development, welche über eine einfache HTML Webseite hinaus gehen. Dazu kommt, dass einige Teammitglieder a
llgemein kaum Programmiererfahrungen hatten, welche über den Inhalt der Vorlesungen "Programmieren" und 
"Softwaretechnik I" hinausgehen. Des Weiteren haben sich dadruch Verzögerungen ergeben, dass niemand aus d
em Team bisher mit den verwendeten Frameworks gearbeitet hat und somit auch keine Erfahrung hatte, wie diese z
u verwenden sind. Genaueres zu den einzelnen Frameworks ist im nachfolgenden Text zu finden. 
Zu Probleme mit den Frameworks kam hinzu, dass wir, was erst im Nachinein aufgefallen ist, in den vorhergehenden P
hasen zu ungenau gearbeitet haben, bzw. die Beschreibungen zu ungenau waren. Dies sorgte für Unklarheiten deren 
Klärung Zeit in Anspruch nahm, welche dafür nicht gedacht war.
Außerdem hat sich herrausgestellt, dass unere Kommunikation innerhalb des Teams fehlerhaft und ungenau war. S
o kam es häufiger vor, dass nicht genau klar war, wer was machen soll oder bis wann etwas fertig sein soll. 
Das wurde durch die erschwerte Terminfindung für physische Treffen noch verstärkt. 





\subsection{hibernate}
Eine sehr große Verzögerung zum eigentlichen Zeitplant hat sich durch \emph{hibernate} ergeben. \emph{Hibernate} wird zur Verwaltung der
Datenbank welche hinter \emph{ChronoCommand} steht verwendet. Die ersten Versuche \empgh{hibernate} zu verwenden wurden dadruch
gehindert, dass noch niemand in unserem Team mit Datenbanken gearbeitet hat. Die ersten Versuche die Dokumentation zu 
\emph{hibernate} verlief sehr erfolglos, da die Dokumentation nicht dafür geeignet ist, sich die Kenntnisse zur Verwendung von
Datenbanken geignet ist. Nach längerem Suchen ergab sich dann, dass \emph{hibernate} eine hibernate.cfg.xml benötigt in welcher 
beschrieben wird, welche Datenbank verwendet wird. Wir haben uns der Einfachheit halber für PostgreSQL entschieden. 


\subsection{view} %TODO jannis und xiaoming

\subsection{Shiro}
\emph{Shiro} dient der Rechte- und Sessionverwaltung von Usern.
Grundlegend gibt es in \emph{Shiro} die Möglichkeit die default Implementierung von \emph{Shiro} zu nutzen, oder mithilfe der Interfaces
eigene Implementierungen und/oder Spezialfällle zu behandeln.
Durch diesen Aufbau gestaltet sich eine Implentation von \emph{Shiros} schwierig, da die Einstiegshilfen wenig abdecken und das Reference Manual
im Gegenzug bei weitem zu ausladend gestaltet ist. Um \emph{Shiro} vollständig zu Implementieren war es in userem Fall nötig, mithilfe der
größeren Coding Samples aus der \emph{Shiro} Source eine funktionierende \emph{Shiro} Implemtierung für unseren Use Case zu schaffen.
Darüber hinaus war zum Zeitpunkt der Implementierung die API der neusten \emph{Shiro} Version nicht verfügbar, was durch die teilweise veralteten Coding Samples
 eine lauffähige Implementierung erschwerte.
 
\subsection{quartz} 
Java \emph{Quartz} ist das Framework, welches uns die Möglichkkeiten geboten hat Prozesse zu einem bestimmten Zeitpunkt auszuführen. Es gab eine Verzögerung im Zeitplan durch die fehlende Kenntnisse im Umgang mit \emph{Quartz} bzw die Implementierung in das Projekt.
Desweiteren hat sich auch ergeben, dass jeder Prozess, den wir über \emph{Quartz} ausführen wollten, eine eigene Klasse benötigt. Als wir aber \emph{Quartz} erfolgreich in das Projekt eingebunden hatten, war es nicht weiter schwierig die gewünschten Funktionen zu programmieren. Somit haben wir monatliche und wöchentliche Events, welche regelmäßig ausgelöst werden und leicht im Code geändert werden können.

\subsection{pdfbox}
Durch die wenigen Gedanken, die wir uns zum Thema \emph{pdfbox} und dem Entwerfen eines Dokumentes, welches man als Stundenzettel benutzen kann, gemacht haben, hat die geplante Zeit für \emph{pdfbox} nicht gereicht. Es war im Vorfeld klar, dass das Generieren eines Stundenzettels als PDF-Dokument, nicht einfach werden würde.
Die Benutzung von \emph{pdfbox} war dann aber doch leichter als erwartet, was uns zugute kam, da wir eine neue Vorlage für den Stundenzettel erstellen mussten. Wir wollten nicht die offizielle KIT-Vorlage als Stundenzettel benutzen, da dort mehr Informationen zum Ausfüllen benötigt wurden als wir erfassen. Es hätte dazu geführt, dass es nach dem Ausdrucken des Stundenzettels noch Felder zum Ausfüllen gegeben hätte, dies wollten wir vermeiden. Die von uns benutzte Vorlage passt genau zu unseren Anforderungen.

%auf Umlaute achten!
