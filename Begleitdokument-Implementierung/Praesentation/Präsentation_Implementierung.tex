%% Einrichtungen für Beamer-Klasse
\documentclass{beamer}
%\mode<presentation>
%{
%	\usetheme{Warsaw}
%	\usecolortheme{whale}
%	\useinnertheme{rectangles}
%	\usefonttheme{structurebold}
%	\useoutertheme{smoothbars}
%}
%\setbeamercovered{transparent}

% Abschalten der kleinen Navigationsleiste am unteren Rand
%\beamertemplatenavigationsymbolsempty   

% Seitenzahl in Fußnote
%\setbeamertemplate{footline}[frame number] 

% Komisches Einblenden deaktivieren
%\setbeamercovered{invisible}

% KIT-Stil
\usepackage{Stylestuff/beamerthemekit}

%% Deutsche Silbentrennung und Beschriftungen
\usepackage[ngerman]{babel}

%% UTF-8-Encoding
\usepackage[utf8]{inputenc}

%% Bibliotheken für viele mathematische Symbole
\usepackage{amsmath, amsfonts, amssymb}

%% Anzeigetiefe für Inhaltsverzeichnis: 1 Stufe
\setcounter{tocdepth}{1}

%% Schönere Schriften
\usepackage[TS1,T1]{fontenc}

%% Tabellen
\usepackage{array}
\usepackage{multicol}

%% Bibliothek für Graphiken
\usepackage{graphicx}

%% der wird sowieso in jeder Datei gesetzt
%%\graphicspath{{../figures/}}

%% Hyperlinks
\usepackage{hyperref}

\usepackage{lmodern}
\usepackage{colortbl}
\usepackage[absolute,overlay]{textpos}
\usepackage{listings}
\usepackage{forloop}
%\usepackage{algorithmic} % PseudoCode package 

\usepackage{tikz}
\usetikzlibrary{matrix}

%%%%%%%%%%%% INHALT %%%%%%%%%%%%%%%%

%% Wochennummer
\newcounter{weeknum}

%% Titelinformationen
\title{Präsentation zur Implementierungsphase}
%\subtitle{Gehalten in den Tutorien Nr. 1, Nr. 12, Nr. 14 und Nr. 29}
\author[Schmid, Friedmann]{\href{mailto:m-pse@minituex.eu}{Maria Schmid}{\href{mailto:jannis@sycoso.eu}{Jannis Friedmann}}} %\href{mailto:dominik.doerner@student.kit.edu}{Dominik Doerner} und \href{mailto:allenouyue@hotmail.com}{Yue Ou}}
\institute{KIT - Karlsruher Institut für Technologie}

%% Titel einfügen
\newcommand{\titleframe}{\frame{\titlepage}}

%% Alles starten mit \starttut{X}
%\newcommand{\starttut}[1]{\part{Tutorium Nr. #1}\setcounter{weeknum}{#1}\titleframe\frame{\frametitle{Inhaltsverzeichnis}\tableofcontents} \AtBeginSection[]{%
%\begin{frame}
%	\tableofcontents{currentsection}
%\end{frame}\addtocounter{framenumber}{-1}

%% Kontakt einfügen
\newcommand{\contact}{\begin{center}Kontakt via E-Mail an \href{mailto:m-pse@minituex.eu}{Maria Schmid}\href{mailto:jannis@sycoso.eu}{Jannis Friedmann}} %\href{mailto:dominik.doerner@student.kit.edu}{Dominik Doerner} oder \href{mailto:allenouyue@hotmail.com}{Yue Ou}\end{center}}

%% Letzte Seite mit Bild []
\newcommand{\lastframe}[3]{\part{end}\frame[plain]{\begin{figure}[H]\centering\includegraphics[scale=#1]{#2}\caption{ \url{#3} }\end{figure} \contact}}

%% Frage für Frage und Antwort Spiel
%\newcommand{\frage}[2]{
%\begin{frame}
%	\frametitle{Was bleibt?}
%		\textit{#1}
%		\newline
%		\hspace*{2em} \only<2>{#2}
%\end{frame}		
%}

%% Wörter
\newcommand{\code}[1]{$\mathbf{#1}$}

%% Sterne

\newcounter{starsc}
\newcommand{\stars}[1]{
	\hfill
	\begin{minipage}{100px}
		\forloop{starsc}{0}{\value{starsc} < #1}%
		{%
			\includegraphics[scale=0.05]{star-full.pdf} \hspace*{1px}
		}%
		\forloop{starsc}{\value{starsc}}{\value{starsc} < 5}%
		{%
			\includegraphics[scale=0.05]{star-empty.pdf} \hspace*{1px}
		}
		\vspace*{2px}
	\end{minipage}
}

%% Verbatim
\usepackage{moreverb}


\include{Praeambel}

\usepackage[table]{xcolor}

\begin{document}

\section{User Interface}
\begin{frame}
	\frametitle{Login}
	\includegraphics{images/login.png}
\end{frame}

\begin{frame}
	\frametitle{Zeiterfassung bearbeiten}
	\includegraphics{images/timerecord-edit.png}
\end{frame}

\begin{frame}
	\frametitle{Nutzerübersicht des Administrators}
	\includegraphics{images/admin-overview.png}
\end{frame}

\begin{frame}
	\frametitle{Neuen Benutzer erstellen}
	\includegraphics{images/create-user.png}
\end{frame}

\begin{frame}
	\frametitle{Stundenzettelübersicht eines Users}
	\includegraphics{images/timesheets-overview.png}
\end{frame}


\section{Probleme}

\subsection{Allgemein}
- Unerfahren
- Frameworks nicht genau genug im voraus angeschaut
- In vorherigen Phasen zu ungenau gearbeitet
-> Wasserfallmodell eher shit
- Interne Schnittstellen nicht richtig definiert und kommuniziert

\subsection{Probleme mit Frameworks}

\begin{frame}
	\frametitle{Probleme mit Frameworks}
	\includegraphics{images/shiro-logo.png}
\end{frame}

\begin{frame}
	\frametitle{Probleme mit Frameworks}
	\includegraphics{images/hibernate-logo.png}
\end{frame}

\begin{frame}
	\frametitle{Probleme mit Frameworks}
	\includegraphics{images/vaadin-logo.png}
\end{frame}

% ggf PDFbox
% ggf Quartz


\section{Kriterien}

\begin{frame}
	\subsection{Musskriterien}
	\begin{tabular}[l]
		Starten und Stoppen von einer zeiterfassung: \emph{erfüllt} \\
		Zeiten nachträglich erfassen: \emph{erfüllt} \\
		Zeiten nachträglich ändern: \emph{erfüllt} \\
		Erfasste Zeiten können Tätigkeiten zugeordnet werden: \emph{erfüllt} \\
		Erfasste Zeiten können Kategorien zugeordnet werden: \emph{erfüllt} \\
		Warnungen wenn Zeiten nicht eingetragen sind: \emph{erfüllt} \\
		Möglichkeit die erfassten Zeiten an den Administrator* und Betreuer* zu übersenden: \emph{erfüllt} \\
		Möglichkeit den Stundenzettel an den Administrator* und Betreuer* zu übersenden: \emph{erfüllt} \\
		Warnungen wenn zu erfassten Zeiten keine Tätigkeit zugeordnet ist: \emph{erfüllt} \\
		Warnungen wenn zu erfassten Zeiten keine Kategorie zugeordnet ist: \emph{erfüllt} \\
		Vergangene Zeiten sind einsehbar: \emph{erfüllt} \\
		Hinzufügen von Accounts: \emph{erfüllt} \\
		Ändern von Accounts: \emph{erfüllt} \\
		Bestimmte Benutzer* sind Administratoren*: \emph{erfüllt} \\
		Bestimmte Benutzer* sind Betreuer*: \emph{erfüllt} \\
		Administratoren* legen fest, welche Benutzer* von welchem Betreuer* betreut werden: \emph{erfüllt} \\
		\cellcolor{pink} Warnungen wenn gesetzlich Pausen genommen werden müssen: \emph{nicht erfüllt, da die Datenerhebung für Pause signifikant komplizierter ist und uns dafür die Zeit nicht mehr gereicht hat} \\
		\cellcolor{pink} Löschen von Accounts: \emph{nicht erfüllt, da keine Accounts gelöscht werden. Nichtbenutze Accounts werden deaktiviert.} \\
		\cellcolor{pink} Backups werden regelmäßig angefertigt: \emph{(nicht) erfüllt, da dies die Aufgabe der Administratoren ist} \\
		\cellcolor{pink} Es kann zwischen LDAP und lokalen Accounts zur Benutzer*verwaltung gewählt werden: \emph{nicht erfüllt, aus Zeitgründen} \\
		\cellcolor{pink} Zeiten sollen durch eine graphische Übersicht visualisiert werden (Heatmap, Punch Card): \emph{nicht erfüllt} \\
	\end{tabular}
\end{frame}

\begin{frame}
	\subsection{Wunschkriterien}
	\begin{tabular}[l]
		Mehrere Frontendimplementierungen sind möglich: \emph{erfüllt}
		Über eine Toolbar Zeiten erfassen und eine Tätigkeit zuordnen: \emph{erfüllt}
		\cellcolor{pink} Zeitvorhersagen anhand vergangener Arbeitszeit: \emph{nicht erfüllt, wegen Zeitmangen und zu wenog Erfahrung mit dem Thema}
		\cellcolor{pink} Überwachung der Arbeitszeit auch ohne abgegebene Zeiten möglich: \emph{nicht erfüllt, da die Umsetzung zu komplex ist} 
		\cellcolor{pink} Erfassung der Tätigkeiten während der Arbeitszeit: \emph{nicht erfüllt, da die Umsetzung zu komplex ist}
		\cellcolor{pink} Möglichkeit, die IP-Range, von der aus sich Administratoren* einloggen können, zu beschränken: \emph{nicht erfüllt, aus Zeitgründen}
		\cellcolor{pink} Kalendarische Übersicht erfasster Zeiten: \emph{nicht erfüllt, aus Zeitgründen}
	\end{tabular}
\end{frame}

\section{Große Veränderungen}


\begin{frame}
	\section{Statistiken}
	\begin{itemize}
		\item Lines of code: 4373
		\item Klassen: 54
		\item Test linecoverage: 21%
		\begin{itemize}
			\item control: 44%
			\item model: 40%
			\item security: 40%
			\item view: 0%
		\end{itemize}
	\end{itemize}
\end{frame}

% add commit statistics?

\end{document}
