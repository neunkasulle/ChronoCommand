%% Einrichtungen für Beamer-Klasse
\documentclass{beamer}
%\mode<presentation>
%{
%	\usetheme{Warsaw}
%	\usecolortheme{whale}
%	\useinnertheme{rectangles}
%	\usefonttheme{structurebold}
%	\useoutertheme{smoothbars}
%}
%\setbeamercovered{transparent}

% Abschalten der kleinen Navigationsleiste am unteren Rand
%\beamertemplatenavigationsymbolsempty   

% Seitenzahl in Fußnote
%\setbeamertemplate{footline}[frame number] 

% Komisches Einblenden deaktivieren
%\setbeamercovered{invisible}

% KIT-Stil
\usepackage{Stylestuff/beamerthemekit}

%% Deutsche Silbentrennung und Beschriftungen
\usepackage[ngerman]{babel}

%% UTF-8-Encoding
\usepackage[utf8]{inputenc}

%% Bibliotheken für viele mathematische Symbole
\usepackage{amsmath, amsfonts, amssymb}

%% Anzeigetiefe für Inhaltsverzeichnis: 1 Stufe
\setcounter{tocdepth}{1}

%% Schönere Schriften
\usepackage[TS1,T1]{fontenc}

%% Tabellen
\usepackage{array}
\usepackage{multicol}

%% Bibliothek für Graphiken
\usepackage{graphicx}

%% der wird sowieso in jeder Datei gesetzt
%%\graphicspath{{../figures/}}

%% Hyperlinks
\usepackage{hyperref}

\usepackage{lmodern}
\usepackage{colortbl}
\usepackage[absolute,overlay]{textpos}
\usepackage{listings}
\usepackage{forloop}
%\usepackage{algorithmic} % PseudoCode package 

\usepackage{tikz}
\usetikzlibrary{matrix}

%%%%%%%%%%%% INHALT %%%%%%%%%%%%%%%%

%% Wochennummer
\newcounter{weeknum}

%% Titelinformationen
\title[Präsentation zur Implementierungsphase]
%\subtitle{Gehalten in den Tutorien Nr. 1, Nr. 12, Nr. 14 und Nr. 29}
\author[Schmid, Friedmann]{\href{mailto:m-pse@minituex.eu}{Maria Schmid}{\href{mailto:jannis@sycoso.eu}{Jannis Friedmann}} %\href{mailto:dominik.doerner@student.kit.edu}{Dominik Doerner} und \href{mailto:allenouyue@hotmail.com}{Yue Ou}}
\institute{KIT - Karlsruher Institut für Technologie}

%% Titel einfügen
\newcommand{\titleframe}{\frame{\titlepage}}

%% Alles starten mit \starttut{X}
%\newcommand{\starttut}[1]{\part{Tutorium Nr. #1}\setcounter{weeknum}{#1}\titleframe\frame{\frametitle{Inhaltsverzeichnis}\tableofcontents} \AtBeginSection[]{%
\begin{frame}
	\tableofcontents[currentsection]
\end{frame}\addtocounter{framenumber}{-1}}}

%% Kontakt einfügen
\newcommand{\contact}{\begin{center}Kontakt via E-Mail an \href{mailto:m-pse@minituex.eu}{Maria Schmid}\href{mailto:jannis@sycoso.eu}{Jannis Friedmann}} %\href{mailto:dominik.doerner@student.kit.edu}{Dominik Doerner} oder \href{mailto:allenouyue@hotmail.com}{Yue Ou}\end{center}}

%% Letzte Seite mit Bild []
\newcommand{\lastframe}[3]{\part{end}\frame[plain]{\begin{figure}[H]\centering\includegraphics[scale=#1]{#2}\caption{ \url{#3} }\end{figure} \contact}}

%% Frage für Frage und Antwort Spiel
%\newcommand{\frage}[2]{
%\begin{frame}
%	\frametitle{Was bleibt?}
%		\textit{#1}
%		\newline
%		\hspace*{2em} \only<2>{#2}
%\end{frame}		
%}

%% Wörter
\newcommand{\code}[1]{$\mathbf{#1}$}

%% Sterne

\newcounter{starsc}
\newcommand{\stars}[1]{
	\hfill
	\begin{minipage}{100px}
		\forloop{starsc}{0}{\value{starsc} < #1}%
		{%
			\includegraphics[scale=0.05]{star-full.pdf} \hspace*{1px}
		}%
		\forloop{starsc}{\value{starsc}}{\value{starsc} < 5}%
		{%
			\includegraphics[scale=0.05]{star-empty.pdf} \hspace*{1px}
		}
		\vspace*{2px}
	\end{minipage}
}

%% Verbatim
\usepackage{moreverb}

